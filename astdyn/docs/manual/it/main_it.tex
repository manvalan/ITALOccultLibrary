\documentclass[12pt,a4paper,twoside]{book}

% ============================================================================
% PACKAGES
% ============================================================================
\usepackage[utf8]{inputenc}
\usepackage[T1]{fontenc}
\usepackage[italian]{babel}
\usepackage{geometry}
\usepackage{graphicx}
\usepackage{amsmath,amssymb,amsthm}
\usepackage{listings}
\usepackage{xcolor}
\usepackage{hyperref}
\usepackage{tikz}
\usepackage{pgfplots}
\usepackage{booktabs}
\usepackage{fancyhdr}
\usepackage{tocloft}
\usepackage{titlesec}
\usepackage{float}

% Font simile a Optima - usando Palatino (eleganza simile)
% pacchetto classico non disponibile, usando alternativa
\usepackage{mathpazo} % Font Palatino (eleganza simile a Optima)
\usepackage[scaled=0.92]{helvet} % Helvetica per sans-serif

% Interlinea 1.3
\usepackage{setspace}
\setstretch{1.3}

% Geometria pagina
\geometry{
    top=3cm,
    bottom=3cm,
    left=3.5cm,
    right=2.5cm,
    headheight=15pt
}

% ============================================================================
% COLORI
% ============================================================================
\definecolor{codebackground}{rgb}{0.95,0.95,0.95}
\definecolor{codekeyword}{rgb}{0.0,0.0,0.5}
\definecolor{codecomment}{rgb}{0.25,0.5,0.35}
\definecolor{codestring}{rgb}{0.6,0.1,0.1}
\definecolor{linkcolor}{rgb}{0.0,0.2,0.6}

% ============================================================================
% HYPERREF
% ============================================================================
\hypersetup{
    colorlinks=true,
    linkcolor=linkcolor,
    citecolor=linkcolor,
    urlcolor=linkcolor,
    bookmarksnumbered=true,
    pdfauthor={Michele Bigi},
    pdftitle={Libreria AstDyn - Manuale Scientifico},
    pdfsubject={Meccanica Celeste e Determinazione Orbitale},
    pdfkeywords={meccanica celeste, determinazione orbitale, astrodinamica}
}

% ============================================================================
% LISTATI CODICE
% ============================================================================
\lstset{
    backgroundcolor=\color{codebackground},
    basicstyle=\ttfamily\small,
    keywordstyle=\color{codekeyword}\bfseries,
    commentstyle=\color{codecomment}\itshape,
    stringstyle=\color{codestring},
    numbers=left,
    numberstyle=\tiny\color{gray},
    stepnumber=1,
    numbersep=8pt,
    showspaces=false,
    showstringspaces=false,
    showtabs=false,
    frame=single,
    rulecolor=\color{black},
    tabsize=4,
    captionpos=b,
    breaklines=true,
    breakatwhitespace=false,
    escapeinside={\%*}{*)},
    xleftmargin=1.5em,
    xrightmargin=0.5em
}

\lstdefinestyle{cpp}{
    language=C++,
    morekeywords={constexpr,nullptr,override,final}
}

% ============================================================================
% TIKZ/PGFPLOTS
% ============================================================================
\usetikzlibrary{shapes,arrows,positioning,calc,decorations.markings}
\pgfplotsset{compat=1.16}

% ============================================================================
% AMBIENTI TEOREMI
% ============================================================================
\theoremstyle{definition}
\newtheorem{definition}{Definizione}[chapter]
\newtheorem{example}{Esempio}[chapter]
\newtheorem{algorithm}{Algoritmo}[chapter]

\theoremstyle{plain}
\newtheorem{theorem}{Teorema}[chapter]
\newtheorem{proposition}{Proposizione}[chapter]
\newtheorem{lemma}{Lemma}[chapter]

\theoremstyle{remark}
\newtheorem{remark}{Nota}[chapter]
\newtheorem{note}{Osservazione}[chapter]

% ============================================================================
% INTESTAZIONI E PIÈ DI PAGINA
% ============================================================================
\pagestyle{fancy}
\fancyhf{}
\fancyhead[LE]{\leftmark}
\fancyhead[RO]{\rightmark}
\fancyfoot[C]{\thepage}
\renewcommand{\headrulewidth}{0.5pt}
\renewcommand{\footrulewidth}{0pt}

% ============================================================================
% FRONTESPIZIO
% ============================================================================
\title{%
    \Huge\textbf{Libreria AstDyn}\\[0.5cm]
    \Large Manuale Scientifico\\[0.3cm]
    \large Versione 1.0.0
}
\author{Michele Bigi}
\date{\today}

% ============================================================================
% DOCUMENTO
% ============================================================================
\begin{document}

% Materia preliminare
\frontmatter
\include{00_frontespizio}
\include{00_prefazione}

\tableofcontents
\listoffigures
\listoftables

% Materia principale
\mainmatter

% Parte I: Fondamenti Teorici
\part{Fondamenti Teorici della Meccanica Celeste}
\include{01_introduzione}
\include{02_sistemi_tempo}
\include{03_sistemi_coordinate}
\include{04_sistemi_riferimento}
\include{05_elementi_orbitali}
\include{06_problema_due_corpi}
\include{07_perturbazioni}

% Parte II: Metodi Numerici
\part{Metodi Numerici e Algoritmi}
\include{08_integrazione_numerica}
\include{09_propagazione_orbite}
\chapter{Matrice di Transizione di Stato}
\label{ch:transizione_stato}

\section{Introduzione}

La \textbf{matrice di transizione di stato} (STM) è fondamentale per la determinazione orbitale, tracciando piccole perturbazioni nel moto orbitale e propagando le incertezze. Questo capitolo sviluppa la teoria matematica e il calcolo pratico della STM.

\section{Fondamenti Matematici}

\subsection{Linearizzazione della Dinamica}

Consideriamo la dinamica orbitale generale:

\begin{equation}
    \dot{\mathbf{y}} = \mathbf{f}(t, \mathbf{y})
\end{equation}

dove $\mathbf{y} = [\mathbf{r}, \mathbf{v}]^T$ è il vettore di stato 6-dimensionale.

Per una traiettoria di riferimento $\mathbf{y}_{\text{rif}}(t)$ e una traiettoria perturbata $\mathbf{y}(t)$, definiamo:

\begin{equation}
    \delta\mathbf{y}(t) = \mathbf{y}(t) - \mathbf{y}_{\text{rif}}(t)
\end{equation}

\subsection{Equazioni Variazionali}

Assumendo piccole perturbazioni, linearizziamo:

\begin{equation}
    \delta\dot{\mathbf{y}} = \frac{\partial\mathbf{f}}{\partial\mathbf{y}}\bigg|_{\mathbf{y}_{\text{rif}}} \delta\mathbf{y} = \mathbf{A}(t) \delta\mathbf{y}
\end{equation}

dove $\mathbf{A}(t)$ è la matrice Jacobiana $6 \times 6$:

\begin{equation}
    \mathbf{A} = \begin{bmatrix}
        \frac{\partial\mathbf{f}_r}{\partial\mathbf{r}} & \frac{\partial\mathbf{f}_r}{\partial\mathbf{v}} \\[8pt]
        \frac{\partial\mathbf{f}_v}{\partial\mathbf{r}} & \frac{\partial\mathbf{f}_v}{\partial\mathbf{v}}
    \end{bmatrix} = \begin{bmatrix}
        \mathbf{0}_{3\times3} & \mathbf{I}_{3\times3} \\[8pt]
        \frac{\partial\mathbf{a}}{\partial\mathbf{r}} & \frac{\partial\mathbf{a}}{\partial\mathbf{v}}
    \end{bmatrix}
\end{equation}

\subsection{Definizione Matrice di Transizione di Stato}

La \textbf{matrice di transizione di stato} $\Phi(t, t_0)$ è la soluzione di:

\begin{equation}
    \frac{d\Phi}{dt} = \mathbf{A}(t)\Phi(t, t_0), \quad \Phi(t_0, t_0) = \mathbf{I}_{6\times6}
\end{equation}

Essa relaziona le perturbazioni di stato a tempi diversi:

\begin{equation}
    \delta\mathbf{y}(t) = \Phi(t, t_0) \delta\mathbf{y}(t_0)
\end{equation}

\subsection{Proprietà}

La STM ha proprietà importanti:

\begin{enumerate}
    \item \textbf{Identità a $t_0$}: $\Phi(t_0, t_0) = \mathbf{I}$
    \item \textbf{Composizione}: $\Phi(t_2, t_0) = \Phi(t_2, t_1)\Phi(t_1, t_0)$
    \item \textbf{Inversa}: $\Phi(t_0, t) = \Phi^{-1}(t, t_0)$
    \item \textbf{Determinante}: $\det[\Phi(t, t_0)] = \exp\left[\int_{t_0}^t \text{tr}(\mathbf{A}(\tau))d\tau\right]$
\end{enumerate}

Per sistemi conservativi (Hamiltoniani), la STM è simplettica: $\Phi^T\mathbf{J}\Phi = \mathbf{J}$ dove $\mathbf{J}$ è la matrice simplettica.

\section{Calcolo della Matrice Jacobiana}

\subsection{Problema dei Due Corpi}

Per il problema kepleriano non perturbato:

\begin{equation}
    \mathbf{a} = -\frac{\mu}{r^3}\mathbf{r}
\end{equation}

Le derivate parziali dell'accelerazione sono:

\begin{equation}
    \frac{\partial\mathbf{a}}{\partial\mathbf{r}} = -\frac{\mu}{r^3}\left[\mathbf{I} - 3\frac{\mathbf{r}\mathbf{r}^T}{r^2}\right]
\end{equation}

\begin{equation}
    \frac{\partial\mathbf{a}}{\partial\mathbf{v}} = \mathbf{0}_{3\times3}
\end{equation}

Quindi:

\begin{equation}
    \mathbf{A}_{\text{2-corpi}} = \begin{bmatrix}
        \mathbf{0} & \mathbf{I} \\[8pt]
        -\frac{\mu}{r^3}\left[\mathbf{I} - 3\frac{\mathbf{r}\mathbf{r}^T}{r^2}\right] & \mathbf{0}
    \end{bmatrix}
\end{equation}

\subsection{Perturbazioni N-Corpi}

Per perturbazioni planetarie, l'accelerazione è:

\begin{equation}
    \mathbf{a}_p = \mu_p \left[\frac{\mathbf{r}_p - \mathbf{r}}{|\mathbf{r}_p - \mathbf{r}|^3} - \frac{\mathbf{r}_p}{r_p^3}\right]
\end{equation}

La derivata parziale rispetto alla posizione:

\begin{equation}
    \frac{\partial\mathbf{a}_p}{\partial\mathbf{r}} = -\frac{\mu_p}{d^3}\left[\mathbf{I} - 3\frac{\mathbf{d}\mathbf{d}^T}{d^2}\right]
\end{equation}

dove $\mathbf{d} = \mathbf{r}_p - \mathbf{r}$ e $d = |\mathbf{d}|$.

\subsection{Correzioni Relativistiche}

L'accelerazione post-Newtoniana include termini dipendenti dalla velocità:

\begin{equation}
    \mathbf{a}_{\text{GR}} = \frac{\mu}{c^2 r^3}\left[4\frac{\mu}{r}\mathbf{r} - v^2\mathbf{r} + 4(\mathbf{r} \cdot \mathbf{v})\mathbf{v}\right]
\end{equation}

Sia $\partial\mathbf{a}_{\text{GR}}/\partial\mathbf{r}$ che $\partial\mathbf{a}_{\text{GR}}/\partial\mathbf{v}$ sono non-zero.

Per la posizione:

\begin{equation}
    \frac{\partial\mathbf{a}_{\text{GR}}}{\partial\mathbf{r}} = \frac{\mu}{c^2 r^3}\left[-v^2\mathbf{I} + 4(\mathbf{v}\mathbf{v}^T) + \text{(termini ordine superiore)}\right]
\end{equation}

Per la velocità:

\begin{equation}
    \frac{\partial\mathbf{a}_{\text{GR}}}{\partial\mathbf{v}} = \frac{\mu}{c^2 r^3}\left[-2v\mathbf{r}\mathbf{v}^T + 4\mathbf{v}\mathbf{r}^T + 4(\mathbf{r} \cdot \mathbf{v})\mathbf{I}\right]
\end{equation}

\subsection{Pressione di Radiazione Solare}

Per SRP con rapporto area-massa costante:

\begin{equation}
    \mathbf{a}_{\text{SRP}} = P_\odot \frac{A}{m} C_R \left(\frac{r_0}{r}\right)^2 \hat{\mathbf{r}}
\end{equation}

La parziale è:

\begin{equation}
    \frac{\partial\mathbf{a}_{\text{SRP}}}{\partial\mathbf{r}} = P_\odot \frac{A}{m} C_R r_0^2 \left[\frac{\mathbf{I}}{r^3} - 3\frac{\mathbf{r}\mathbf{r}^T}{r^5}\right]
\end{equation}

\section{Calcolo Numerico}

\subsection{Vettore di Stato Aumentato}

Per calcolare la STM numericamente, aumentiamo il vettore di stato:

\begin{equation}
    \tilde{\mathbf{y}} = \begin{bmatrix} \mathbf{y} \\ \text{vec}(\Phi) \end{bmatrix} \in \mathbb{R}^{42}
\end{equation}

dove $\text{vec}(\Phi)$ impila i 36 elementi di $\Phi$ per colonne.

\subsection{Dinamica Aumentata}

Il sistema aumentato è:

\begin{equation}
    \frac{d\tilde{\mathbf{y}}}{dt} = \begin{bmatrix} \mathbf{f}(\mathbf{y}) \\ \text{vec}(\mathbf{A}(\mathbf{y})\Phi) \end{bmatrix}
\end{equation}

In pratica, integriamo:
\begin{itemize}
    \item 6 equazioni per lo stato $\mathbf{y}$
    \item 36 equazioni per gli elementi STM
    \item Totale: 42 ODE accoppiate
\end{itemize}

\subsection{Implementazione in AstDyn}

\begin{lstlisting}[language=C++,caption={Propagazione STM con STMPropagator}]
#include "astdyn/propagation/STMPropagator.hpp"
#include "astdyn/propagation/AnalyticalJacobian.hpp"

using namespace astdyn::propagation;

// 1. Configurazione Funzione Forza (es. 2-corpi)
double mu = 1.327e11; // GM Sole
auto force_func = [mu](double t, const Vector6d& y) {
    Vector3d r = y.head<3>();
    double r_norm = r.norm();
    Vector3d acc = -mu * r / (r_norm * r_norm * r_norm);
    Vector6d dydt;
    dydt.head<3>() = y.tail<3>();
    dydt.tail<3>() = acc;
    return dydt;
};

// 2. Configurazione Jacobiano (Analitico è più veloce/preciso)
auto jac_func = [mu](double t, const Vector6d& y) {
    return AnalyticalJacobian::two_body(y, mu);
};

// 3. Istanziare STMPropagator
auto integrator = std::make_unique<RKF78Integrator>(0.1, 1e-12);
STMPropagator stm_prop(std::move(integrator), force_func, jac_func);

// 4. Propagare
Vector6d y0 = ...; // Stato iniziale
double t0 = 60000.0;
double tf = 60100.0;

auto result = stm_prop.propagate(y0, t0, tf);

Vector6d yf = result.state;
Matrix6d Phi = result.stm;

std::cout << "Determinante STM: " << Phi.determinant() << "\n";
\end{lstlisting}

\subsection{Costo Computazionale}

Il calcolo STM aumenta il costo computazionale:

\begin{table}[H]
\centering
\begin{tabular}{lcc}
\toprule
\textbf{Calcolo} & \textbf{Equazioni Stato} & \textbf{Fattore Tempo CPU} \\
\midrule
Solo stato & 6 & 1.0$\times$ \\
Stato + STM & 42 & 5-7$\times$ \\
Stato + STM + sensibilità & 42 + 6$N_p$ & 10-15$\times$ \\
\bottomrule
\end{tabular}
\caption{Costo computazionale propagazione STM. $N_p$ è il numero di parametri.}
\label{tab:costo_stm}
\end{table}

\section{Applicazioni}

\subsection{Determinazione Orbitale}

Nella correzione differenziale (fit orbitale minimi quadrati), necessitiamo:

\begin{equation}
    \frac{\partial\mathbf{y}(t_{\text{oss}})}{\partial\mathbf{y}(t_0)} = \Phi(t_{\text{oss}}, t_0)
\end{equation}

Questo relaziona le osservazioni alle condizioni iniziali, abilitando il raffinamento orbitale iterativo.

\subsection{Propagazione Covarianza}

Data la covarianza iniziale $\mathbf{P}_0$, la covarianza al tempo $t$ è:

\begin{equation}
    \mathbf{P}(t) = \Phi(t, t_0) \mathbf{P}_0 \Phi^T(t, t_0)
\end{equation}

Questo quantifica la crescita dell'incertezza nel tempo.

Esempio:
\begin{lstlisting}[language=C++,caption={Propagazione covarianza}]
Matrix6d P0 = initial_covariance();  // km^2, (km/s)^2
Matrix6d Phi = result.stm;

Matrix6d Pf = Phi * P0 * Phi.transpose();

// Incertezza posizione al tempo finale
Vector3d sigma_pos = Pf.block<3,3>(0,0).diagonal().cwiseSqrt();
std::cout << "Incertezza posizione: " 
          << sigma_pos.transpose() << " km\n";
\end{lstlisting}

\subsection{Analisi di Sensibilità}

La STM rivela come le perturbazioni nelle condizioni iniziali influenzano gli stati futuri:

\begin{equation}
    \frac{\partial r(t)}{\partial r_0} = \Phi_{11}(t, t_0), \quad
    \frac{\partial r(t)}{\partial v_0} = \Phi_{12}(t, t_0)
\end{equation}

Questi sono i blocchi $3 \times 3$ superiore-sinistro e superiore-destro di $\Phi$.

\subsection{Ottimizzazione Manovre}

Per la progettazione di traiettorie spaziali, la STM aiuta a calcolare:
\begin{itemize}
    \item Matrici di puntamento (dove mirare per colpire un bersaglio)
    \item Requisiti $\Delta v$
    \item Sensibilità a errori di esecuzione
\end{itemize}

\section{STM Analitica vs Numerica}

\subsection{STM Analitica per Moto Kepleriano}

Per il problema non perturbato dei due corpi, esistono soluzioni in forma chiusa. La STM può essere espressa in termini di elementi orbitali e loro derivate.

Vantaggi:
\begin{itemize}
    \item Esatta (nessun errore numerico)
    \item Veloce da valutare
    \item Valida per lunghi intervalli temporali
\end{itemize}

Svantaggi:
\begin{itemize}
    \item Formule complesse (specialmente vicino a singolarità)
    \item Non include perturbazioni
    \item Uso pratico limitato
\end{itemize}

\subsection{STM Numerica}

Integrando numericamente le equazioni variazionali:

Vantaggi:
\begin{itemize}
    \item Gestisce modelli di forza arbitrari
    \item Implementazione diretta
    \item Include tutte le perturbazioni
\end{itemize}

Svantaggi:
\begin{itemize}
    \item Accumulo errore numerico
    \item 7$\times$ più lenta della propagazione solo-stato
    \item Mal condizionamento per archi lunghi
\end{itemize}

\subsection{Approcci Ibridi}

Per alcune applicazioni, usare:
\begin{enumerate}
    \item STM analitica per parte kepleriana
    \item Correzioni perturbazione numeriche
    \item Composizione transizione stato
\end{enumerate}

\section{Stabilità Numerica}

\subsection{Problemi di Condizionamento}

La STM diventa mal condizionata per:
\begin{itemize}
    \item Tempi di propagazione lunghi ($>$ diversi periodi orbitali)
    \item Orbite ad alta eccentricità
    \item Moto quasi-rettilineo
\end{itemize}

Crescita del numero di condizione:

\begin{equation}
    \kappa(\Phi) \approx \exp\left(\lambda_{\max} \Delta t\right)
\end{equation}

dove $\lambda_{\max}$ è il più grande esponente di Lyapunov.

\subsection{Strategie di Mitigazione}

\textbf{1. Rilinearizzazione}

Invece di propagare da $t_0$ a $t_f$, dividere in segmenti:

\begin{equation}
    \Phi(t_f, t_0) = \Phi(t_f, t_2) \Phi(t_2, t_1) \Phi(t_1, t_0)
\end{equation}

Ogni segmento ha migliore condizionamento.

\textbf{2. Transizione stato in elementi orbitali}

Invece della STM cartesiana, usare:

\begin{equation}
    \frac{\partial\mathbf{e}(t)}{\partial\mathbf{e}(t_0)}
\end{equation}

dove $\mathbf{e} = [a, e, i, \Omega, \omega, M]$ sono elementi orbitali.

\textbf{3. Regolarizzazione}

Usare coordinate regolarizzate (Kustaanheimo-Stiefel, Sperling-Burdet) che si comportano meglio vicino al periasse.

\section{Esempio Pratico}

\subsection{Tracciamento Bersaglio}

Tracciare l'incertezza nella posizione asteroidale per valutazione impatto:

\begin{lstlisting}[language=C++,caption={Propagazione incertezza asteroide}]
// Stato iniziale da determinazione orbitale
Vector6d y0 = {1.1, 0.2, 0.05, -0.01, 0.03, 0.0};  // AU, AU/giorno

// Covarianza iniziale (da fit minimi quadrati)
Matrix6d P0 = Matrix6d::Zero();
P0.diagonal() << 1e-8, 1e-8, 1e-9,  // pos: 1500 km
                 1e-11, 1e-11, 1e-12;  // vel: 0.15 m/s

ForceModel forces;
forces.enable_planets({"Earth", "Jupiter", "Venus", "Mars"});

Propagator prop(forces);
prop.enable_stm(true);

// Propaga 10 anni
double t0 = 60000.0;
double tf = t0 + 3652.5;  // 10 anni

auto result = prop.propagate_with_stm(y0, t0, tf);

// Calcola incertezza al tempo futuro
Matrix6d Pf = result.stm * P0 * result.stm.transpose();

// Incertezza posizione (3-sigma)
Vector3d sigma_3 = 3.0 * Pf.block<3,3>(0,0).diagonal().cwiseSqrt();
std::cout << "Incertezza posizione (3-sigma): \n";
std::cout << sigma_3.transpose() * AU_TO_KM << " km\n";

// Controlla avvicinamento ravvicinato Terra
Vector6d earth_state = ephemeris.get_planet("Earth", tf);
Vector3d rel_pos = result.state.head<3>() - earth_state.head<3>();
double distance = rel_pos.norm() * AU_TO_KM;

std::cout << "Distanza dalla Terra: " << distance << " km\n";
std::cout << "Probabilita' impatto (Gaussiana): ";
if (distance < 3.0 * sigma_3.norm() * AU_TO_KM) {
    std::cout << "NON-ZERO - richiesta ulteriore analisi\n";
} else {
    std::cout << "Trascurabile\n";
}
\end{lstlisting}

\subsection{Pianificazione Osservazioni}

Determinare tempi ottimali di osservazione per ridurre incertezza:

\begin{lstlisting}[language=C++,caption={Pianificazione osservazioni}]
// Propaga con STM a epoche osservative multiple
std::vector<double> obs_times = {t0 + 30, t0 + 60, t0 + 90};

for (double t_obs : obs_times) {
    auto result = prop.propagate_with_stm(y0, t0, t_obs);
    Matrix6d P = result.stm * P0 * result.stm.transpose();
    
    // Incertezza RA/Dec da incertezza posizione
    Vector3d r = result.state.head<3>();
    double dec = std::asin(r(2) / r.norm());
    double ra = std::atan2(r(1), r(0));
    
    // Approssimazione semplice (calcolo completo usa parziali osservazione)
    double sigma_ra = P(0,0) / (r.norm() * std::cos(dec));
    double sigma_dec = P(2,2) / r.norm();
    
    std::cout << "Epoca " << t_obs << ": "
              << "sigma_RA = " << sigma_ra * RAD_TO_ARCSEC << " arcosec, "
              << "sigma_Dec = " << sigma_dec * RAD_TO_ARCSEC << " arcosec\n";
}
\end{lstlisting}

\section{Sensibilità Parametri}

\subsection{Vettore di Stato Esteso}

Per studiare la sensibilità a parametri dinamici (es. $\mu$, $C_R$, masse asteroidi), aumentare lo stato:

\begin{equation}
    \tilde{\mathbf{y}} = \begin{bmatrix} \mathbf{y} \\ \mathbf{p} \end{bmatrix}
\end{equation}

dove $\mathbf{p}$ sono parametri. Quindi:

\begin{equation}
    \frac{d}{dt}\begin{bmatrix} \mathbf{y} \\ \mathbf{p} \end{bmatrix} = \begin{bmatrix} \mathbf{f}(\mathbf{y}, \mathbf{p}) \\ \mathbf{0} \end{bmatrix}
\end{equation}

La STM estesa include $\partial\mathbf{y}/\partial\mathbf{p}$.

\subsection{Matrici di Sensibilità}

Definire matrice di sensibilità:

\begin{equation}
    \mathbf{S}(t) = \frac{\partial\mathbf{y}(t)}{\partial\mathbf{p}}
\end{equation}

Essa soddisfa:

\begin{equation}
    \frac{d\mathbf{S}}{dt} = \mathbf{A}(t)\mathbf{S} + \frac{\partial\mathbf{f}}{\partial\mathbf{p}}
\end{equation}

Questo rivela come il moto orbitale dipende dai parametri fisici.

\section{Riepilogo}

Concetti chiave sulla matrice di transizione di stato:

\begin{enumerate}
    \item La \textbf{STM} $\Phi(t, t_0)$ propaga linearmente piccole perturbazioni
    \item Soddisfa le \textbf{equazioni variazionali}: $\dot{\Phi} = \mathbf{A}(t)\Phi$
    \item La \textbf{matrice Jacobiana} $\mathbf{A}$ contiene derivate del modello di forza
    \item Il \textbf{calcolo numerico} richiede integrazione di 42 ODE (6 stato + 36 STM)
    \item \textbf{Applicazioni}: determinazione orbitale, propagazione covarianza, analisi sensibilità
    \item Il \textbf{condizionamento} degrada per archi lunghi; usare rilinearizzazione
    \item La \textbf{STM estesa} include sensibilità parametri
\end{enumerate}

Comprendere la STM è essenziale per:
\begin{itemize}
    \item Determinazione orbitale precisa (Capitolo 14)
    \item Quantificazione incertezza
    \item Progettazione missioni e puntamento
    \item Stima parametri
    \item Valutazione probabilità impatto
\end{itemize}

Il prossimo capitolo copre il calcolo di effemeridi e metodi di interpolazione per ricerca efficiente dello stato.

\include{11_effemeridi}

% Parte III: Determinazione Orbitale
\part{Determinazione Orbitale}
\include{12_osservazioni}
\include{13_orbita_iniziale}
\chapter{Correzione Differenziale}
\label{ch:differential_correction}

\section{Introduzione}

La \textbf{correzione differenziale} (Differential Correction, DC) è il raffinamento iterativo ai minimi quadrati di un'orbita usando tutte le osservazioni disponibili. È la pietra angolare della determinazione orbitale.

\textbf{Input}: Orbita iniziale + osservazioni

\textbf{Output}: Orbita migliorata + matrice di covarianza + residui

\textbf{Metodo}: Minimi quadrati pesati minimizzando i residui O-C (osservato meno calcolato).

\section{Il Problema dei Minimi Quadrati}

\subsection{Equazione di Osservazione}

Per l'osservazione $i$:

\begin{equation}
    \mathbf{o}_i = \mathbf{h}(\mathbf{y}_0, t_i) + \boldsymbol{\epsilon}_i
\end{equation}

dove:
\begin{itemize}
    \item $\mathbf{o}_i$: Valore osservato (es., RA, Dec)
    \item $\mathbf{h}$: Modello di osservazione (trasformazione coordinate)
    \item $\mathbf{y}_0$: Stato all'epoca $t_0$
    \item $\boldsymbol{\epsilon}_i \sim \mathcal{N}(0, \mathbf{W}_i^{-1})$: Errore di misura
\end{itemize}

\subsection{Linearizzazione}

Linearizzare attorno alla stima corrente $\mathbf{y}_0^{(k)}$:

\begin{equation}
    \mathbf{o}_i - \mathbf{c}_i = \mathbf{H}_i \Delta\mathbf{y}_0 + \boldsymbol{\epsilon}_i
\end{equation}

dove:
\begin{itemize}
    \item $\mathbf{c}_i = \mathbf{h}(\mathbf{y}_0^{(k)}, t_i)$: Valore calcolato
    \item $\mathbf{H}_i = \frac{\partial \mathbf{h}}{\partial \mathbf{y}_0}$: Matrice di disegno (derivate parziali osservazione)
    \item $\Delta\mathbf{y}_0 = \mathbf{y}_0 - \mathbf{y}_0^{(k)}$: Correzione allo stato
\end{itemize}

\subsection{Equazioni Normali}

Minimizzare la somma pesata dei quadrati dei residui:

\begin{equation}
    \chi^2 = \sum_{i=1}^m (\mathbf{o}_i - \mathbf{c}_i - \mathbf{H}_i \Delta\mathbf{y}_0)^T \mathbf{W}_i (\mathbf{o}_i - \mathbf{c}_i - \mathbf{H}_i \Delta\mathbf{y}_0)
\end{equation}

Soluzione:

\begin{equation}
    (\mathbf{H}^T \mathbf{W} \mathbf{H}) \Delta\mathbf{y}_0 = \mathbf{H}^T \mathbf{W} (\mathbf{o} - \mathbf{c})
\end{equation}

Definire:
\begin{align}
    \mathbf{N} &= \mathbf{H}^T \mathbf{W} \mathbf{H} \quad \text{(matrice normale)} \\
    \mathbf{b} &= \mathbf{H}^T \mathbf{W} (\mathbf{o} - \mathbf{c}) \quad \text{(termine noto)}
\end{align}

Soluzione: $\mathbf{N} \Delta\mathbf{y}_0 = \mathbf{b}$

Covarianza: $\mathbf{C} = \mathbf{N}^{-1}$

\section{Calcolo delle Derivate Parziali}

\subsection{Regola della Catena con STM}

Per osservazioni RA/Dec al tempo $t_i$:

\begin{equation}
    \mathbf{H}_i = \frac{\partial (\alpha, \delta)}{\partial \mathbf{y}_0} = \frac{\partial (\alpha, \delta)}{\partial \mathbf{y}(t_i)} \frac{\partial \mathbf{y}(t_i)}{\partial \mathbf{y}_0}
\end{equation}

dove $\Phi(t_i, t_0) = \frac{\partial \mathbf{y}(t_i)}{\partial \mathbf{y}_0}$ è la matrice di transizione di stato (Capitolo 10).

\subsection{Derivate Geometriche}

Dalla posizione topocentrica $\boldsymbol{\rho} = \mathbf{r} - \mathbf{R}$:

\begin{align}
    \alpha &= \arctan2(\rho_y, \rho_x) \\
    \delta &= \arcsin(\rho_z / \rho)
\end{align}

Derivate:

\begin{align}
    \frac{\partial \alpha}{\partial \rho_x} &= -\frac{\rho_y}{\rho_x^2 + \rho_y^2} \\
    \frac{\partial \alpha}{\partial \rho_y} &= \frac{\rho_x}{\rho_x^2 + \rho_y^2} \\
    \frac{\partial \alpha}{\partial \rho_z} &= 0 \\
    \frac{\partial \delta}{\partial \rho_x} &= -\frac{\rho_x \rho_z}{\rho^2 \sqrt{\rho_x^2 + \rho_y^2}} \\
    \frac{\partial \delta}{\partial \rho_y} &= -\frac{\rho_y \rho_z}{\rho^2 \sqrt{\rho_x^2 + \rho_y^2}} \\
    \frac{\partial \delta}{\partial \rho_z} &= \frac{\sqrt{\rho_x^2 + \rho_y^2}}{\rho^2}
\end{align}

\subsection{Rotazione del Sistema di Riferimento}

Un aspetto critico dell'implementazione pratica è la gestione dei diversi sistemi di coordinate.
Tipicamente, l'integrazione numerica e la propagazione STM sono eseguite nel riferimento **Eliocentrico Eclittico J2000** (per allinearsi con le effemeridi planetarie come VSOP87), mentre le osservazioni sono riportate nel riferimento **Topocentrico Equatoriale J2000** (RA/Dec).

Pertanto, la regola della catena deve includere una matrice di rotazione $\mathbf{R}_{\text{ecl}\to\text{eq}}$:

\begin{equation}
    \mathbf{H}_i = \frac{\partial (\alpha, \delta)}{\partial \mathbf{r}_{\text{eq}}} \cdot \mathbf{R}_{\text{ecl}\to\text{eq}} \cdot \Phi_{\text{ecl}}(t_i, t_0)
\end{equation}

dove:
\begin{itemize}
    \item $\frac{\partial (\alpha, \delta)}{\partial \mathbf{r}_{\text{eq}}}$ sono le derivate geometriche nel riferimento equatoriale.
    \item $\mathbf{R}_{\text{ecl}\to\text{eq}}$ è la matrice di rotazione per l'obliquità dell'eclittica ($\epsilon \approx 23.44^\circ$).
    \item $\Phi_{\text{ecl}}(t_i, t_0)$ è la STM nel riferimento eclittico.
\end{itemize}

Trascurare questa rotazione durante il calcolo delle derivate parziali porterà alla divergenza del fit, poiché la direzione del gradiente sarà errata.

\subsection{Correzione per il Tempo Luce}

Lo stato $\mathbf{y}(t_i)$ usato nell'equazione di osservazione è in realtà lo stato al tempo ritardato $t_i - \tau$, dove $\tau$ è il tempo di viaggio della luce. Le derivate parziali dovrebbero tecnicamente tenere conto di questo spostamento temporale, ma per asteroidi della fascia principale, l'approssimazione $\frac{\partial \mathbf{y}(t_i-\tau)}{\partial \mathbf{y}_0} \approx \Phi(t_i, t_0)$ è solitamente sufficiente.

\subsection{Derivate Complete}

Combinare le derivate geometriche, la rotazione e $\Phi$:

\begin{equation}
    \mathbf{H}_i = \begin{bmatrix}
        \frac{\partial \alpha}{\partial x_{\text{eq}}} & \frac{\partial \alpha}{\partial y_{\text{eq}}} & \frac{\partial \alpha}{\partial z_{\text{eq}}} & 0 & 0 & 0 \\
        \frac{\partial \delta}{\partial x_{\text{eq}}} & \frac{\partial \delta}{\partial y_{\text{eq}}} & \frac{\partial \delta}{\partial z_{\text{eq}}} & 0 & 0 & 0
    \end{bmatrix} 
    \begin{bmatrix}
        \mathbf{R} & \mathbf{0} \\
        \mathbf{0} & \mathbf{R}
    \end{bmatrix}
    \Phi_{\text{ecl}}(t_i, t_0)
\end{equation}

Nota: RA/Dec dipendono solo dalla posizione, non dalla velocità, nelle derivate geometriche. La velocità influenza le osservazioni attraverso la propagazione ($\Phi$).

\section{Algoritmo}

\textbf{Input}: Orbita iniziale $\mathbf{y}_0^{(0)}$, osservazioni $\{(\mathbf{o}_i, t_i, \mathbf{W}_i)\}$

\textbf{Iterare}:
\begin{enumerate}
    \item Per ogni osservazione $i$:
    \begin{enumerate}
        \item Propagare a $t_i$ con STM: $[\mathbf{y}(t_i), \Phi(t_i, t_0)]$
        \item Calcolare predizione $\mathbf{c}_i = \mathbf{h}(\mathbf{y}(t_i))$
        \item Calcolare derivate geometriche
        \item Calcolare derivate complete $\mathbf{H}_i$ usando STM
    \end{enumerate}
    \item Formare matrice normale: $\mathbf{N} = \sum_i \mathbf{H}_i^T \mathbf{W}_i \mathbf{H}_i$
    \item Formare termine noto: $\mathbf{b} = \sum_i \mathbf{H}_i^T \mathbf{W}_i (\mathbf{o}_i - \mathbf{c}_i)$
    \item Risolvere: $\mathbf{N} \Delta\mathbf{y}_0 = \mathbf{b}$
    \item Aggiornare: $\mathbf{y}_0^{(k+1)} = \mathbf{y}_0^{(k)} + \Delta\mathbf{y}_0$
    \item Calcolare RMS: $\text{RMS} = \sqrt{\frac{1}{m-n} \sum_i w_i r_i^2}$ dove $r_i = \mathbf{o}_i - \mathbf{c}_i$
    \item Verificare convergenza: $|\Delta\mathbf{y}_0| < \epsilon$ e $|\Delta\text{RMS}| < \epsilon_{\text{RMS}}$
\end{enumerate}

\textbf{Output}: Stato converso $\mathbf{y}_0^*$, covarianza $\mathbf{C} = \mathbf{N}^{-1}$, residui

\section{Criteri di Convergenza}

\subsection{Correzione allo Stato}

\begin{equation}
    ||\Delta\mathbf{y}_0|| < 10^{-8} \text{ AU, AU/giorno}
\end{equation}

\subsection{Variazione RMS}

\begin{equation}
    \frac{|\text{RMS}^{(k+1)} - \text{RMS}^{(k)}|}{\text{RMS}^{(k)}} < 10^{-6}
\end{equation}

\subsection{Iterazioni Massime}

Converge tipicamente in 3-10 iterazioni. Se non converge dopo 20 iterazioni, sospettare:
\begin{itemize}
    \item Orbita iniziale scadente
    \item Osservazioni errate (outlier)
    \item Modello inadeguato (perturbazioni mancanti)
\end{itemize}

\section{Strategia di Pesatura}

\subsection{Pesi Empirici}

Per osservazioni RA/Dec:

\begin{equation}
    w_{\alpha,i} = \frac{1}{\sigma_{\alpha,i}^2}, \quad w_{\delta,i} = \frac{1}{\sigma_{\delta,i}^2}
\end{equation}

$\sigma$ tipici:
\begin{itemize}
    \item CCD moderno (calibrato Gaia): 0.1"
    \item CCD amatoriale: 0.5"
    \item Fotografico storico: 1-2"
\end{itemize}

\subsection{Pesatura Robusta}

Ridurre peso degli outlier usando pesi di Huber:

\begin{equation}
    w_i' = \begin{cases}
        w_i & \text{se } |r_i| < k\sigma \\
        w_i \frac{k\sigma}{|r_i|} & \text{se } |r_i| \ge k\sigma
    \end{cases}
\end{equation}

dove $k = 2.5$ (tipico).

\section{Matrice di Covarianza}

\subsection{Incertezza Formale}

Dalla matrice normale:

\begin{equation}
    \mathbf{C} = \mathbf{N}^{-1} = (\mathbf{H}^T \mathbf{W} \mathbf{H})^{-1}
\end{equation}

Elementi diagonali: $\sigma_i = \sqrt{C_{ii}}$

\textbf{Esempio} (asteroide con 100 osservazioni su 30 giorni):
\begin{itemize}
    \item $\sigma_x \sim 10^{-7}$ AU (15 km)
    \item $\sigma_v \sim 10^{-9}$ AU/giorno (1.7 mm/s)
\end{itemize}

\subsection{Correlazione}

Gli elementi fuori diagonale mostrano le correlazioni tra parametri:

\begin{equation}
    \rho_{ij} = \frac{C_{ij}}{\sqrt{C_{ii} C_{jj}}}
\end{equation}

Correlazioni forti (es., $\rho_{xy} > 0.9$) indicano problemi di geometria osservativa.

\subsection{Incertezza Propagata}

Al tempo $t$:

\begin{equation}
    \mathbf{C}(t) = \Phi(t, t_0) \mathbf{C}(t_0) \Phi(t, t_0)^T
\end{equation}

L'incertezza cresce con il tempo. Per soluzioni ad arco corto, $\sigma$ può aumentare esponenzialmente.

\section{Implementazione}

\begin{lstlisting}[language=C++,caption={Implementazione della correzione differenziale}]
struct DCResult {
    Vector6d state;
    Matrix6d covariance;
    double rms;
    int iterations;
    std::vector<double> residuals;
};

DCResult differential_correction(
    const Vector6d& initial_state,
    double epoch,
    const std::vector<Observation>& observations,
    const ForceModel& forces,
    const EphemerisInterface& ephemeris,
    int max_iterations = 20,
    double tol = 1e-8)
{
    Vector6d y0 = initial_state;
    double prev_rms = 1e10;
    
    for (int iter = 0; iter < max_iterations; ++iter) {
        // Accumulare matrice normale e termine noto
        Matrix6d N = Matrix6d::Zero();
        Vector6d b = Vector6d::Zero();
        double chi2 = 0.0;
        std::vector<double> residuals;
        
        for (const auto& obs : observations) {
            // Propagare con STM
            auto [y_obs, Phi] = propagate_with_stm(y0, epoch, obs.epoch, forces);
            
            // Predire osservazione
            Vector2d computed = predict_observation(y_obs, obs.epoch, obs.obs_code, ephemeris);
            
            // Residuo (O-C)
            Vector2d residual;
            residual(0) = (obs.ra - computed(0)) * cos(obs.dec);  // RA cos(Dec)
            residual(1) = obs.dec - computed(1);  // Dec
            
            residuals.push_back(residual.norm() * RAD_TO_ARCSEC);
            
            // Derivate geometriche
            Matrix<double, 2, 3> geom_partials = compute_ra_dec_partials(y_obs, obs, ephemeris);
            
            // Derivate complete via STM
            Matrix<double, 2, 6> H;
            H.block<2, 3>(0, 0) = geom_partials;
            H.block<2, 3>(0, 3).setZero();
            H = H * Phi;  // Regola della catena
            
            // Pesi
            double w_ra = 1.0 / (obs.sigma_ra * obs.sigma_ra);
            double w_dec = 1.0 / (obs.sigma_dec * obs.sigma_dec);
            Matrix2d W = Vector2d(w_ra, w_dec).asDiagonal();
            
            // Accumulare equazioni normali
            N += H.transpose() * W * H;
            b += H.transpose() * W * residual;
            chi2 += residual.transpose() * W * residual;
        }
        
        // Risolvere equazioni normali
        Vector6d delta_y0 = N.ldlt().solve(b);
        
        // Aggiornare stato
        y0 += delta_y0;
        
        // Calcolare RMS
        int dof = 2 * observations.size() - 6;  // gradi di liberta'
        double rms = sqrt(chi2 / dof) * RAD_TO_ARCSEC;
        
        // Verificare convergenza
        if (delta_y0.norm() < tol && abs(rms - prev_rms) < 1e-6) {
            Matrix6d covariance = N.inverse();
            return {y0, covariance, rms, iter + 1, residuals};
        }
        
        prev_rms = rms;
    }
    
    throw std::runtime_error("DC non convergente");
}
\end{lstlisting}

\section{Esempio: Asteroide 203 Pompeja}

\subsection{Definizione del Problema}

\begin{itemize}
    \item Oggetto: 203 Pompeja (asteroide della Fascia Principale)
    \item Osservazioni: 100 misure RA/Dec
    \item Arco temporale: 60 giorni
    \item Osservatorio: 500 (geocentrico), F51 (Pan-STARRS)
    \item Orbita iniziale: Da JPL Horizons
\end{itemize}

\subsection{Risultati}

\begin{lstlisting}[language=C++,caption={Esecuzione DC su Pompeja}]
// Caricare osservazioni da file formato MPC
std::vector<Observation> obs = load_mpc_observations("pompeja.obs");
std::cout << "Caricate " << obs.size() << " osservazioni\n";

// Orbita iniziale da Horizons
Vector6d y0_initial = /* ... da JPL ... */;
double epoch = 2460000.5;  // JD

// Modello di forze
auto forces = std::make_shared<ForceModel>();
forces->add_perturbation(std::make_shared<SunGravity>());
forces->add_perturbation(std::make_shared<JupiterPerturbation>());
forces->add_perturbation(std::make_shared<SaturnPerturbation>());

// Effemeridi
SpiceInterface spice;
spice.load_kernel("de440.bsp");

// Eseguire correzione differenziale
try {
    auto result = differential_correction(y0_initial, epoch, obs, *forces, spice);
    
    std::cout << "Convergenza in " << result.iterations << " iterazioni\n";
    std::cout << "RMS = " << result.rms << " arcosec\n";
    
    // Stampare elementi orbitali
    OrbitalElements elem = OrbitalElements::from_cartesian(result.state, epoch);
    std::cout << "\nOrbita migliorata:\n";
    std::cout << "a = " << elem.a << " +/- " << sqrt(result.covariance(0,0)) << " AU\n";
    std::cout << "e = " << elem.e << " +/- " << sqrt(result.covariance(1,1)) << "\n";
    std::cout << "i = " << elem.i * RAD_TO_DEG << " deg\n";
    
    // Residui maggiori
    std::sort(result.residuals.begin(), result.residuals.end(), std::greater<>());
    std::cout << "\nTop 5 residui:\n";
    for (int i = 0; i < 5; ++i) {
        std::cout << i+1 << ". " << result.residuals[i] << " arcosec\n";
    }
    
} catch (const std::exception& e) {
    std::cerr << "Errore: " << e.what() << "\n";
}
\end{lstlisting}

\textbf{Output tipico}:
\begin{verbatim}
Caricate 100 osservazioni
Convergenza in 5 iterazioni
RMS = 0.658 arcosec

Orbita migliorata:
a = 2.7436 +/- 0.000001 AU
e = 0.0624 +/- 0.000005
i = 11.743 deg

Top 5 residui:
1. 2.34 arcosec
2. 1.98 arcosec
3. 1.76 arcosec
4. 1.65 arcosec
5. 1.54 arcosec
\end{verbatim}

\subsection{Interpretazione}

\begin{itemize}
    \item \textbf{RMS = 0.658"}: Eccellente adattamento, coerente con precisione astrometria CCD
    \item \textbf{5 iterazioni}: Convergenza rapida indica buona orbita iniziale
    \item \textbf{$\sigma_a = 10^{-6}$ AU}: Semiasse maggiore determinato a ~150 km
    \item \textbf{Residui maggiori $<$2.5"}: Nessun outlier ovvio
    \item \textbf{Covarianza}: Incertezza formale, propagare per errore effemeridi
\end{itemize}

\section{Risoluzione Problemi}

\subsection{Non-Convergenza}

\textbf{Sintomi}: RMS oscilla o aumenta.

\textbf{Cause}:
\begin{enumerate}
    \item Orbita iniziale scadente (troppo lontana dalla verità)
    \item Outlier che dominano il fit
    \item Modello di forze inadeguato
    \item Problemi numerici (matrice normale mal condizionata)
\end{enumerate}

\textbf{Soluzioni}:
\begin{itemize}
    \item Migliorare IOD
    \item Abilitare pesatura robusta
    \item Aggiungere perturbazioni mancanti
    \item Regolarizzare matrice normale
\end{itemize}

\subsection{RMS Elevato}

\textbf{Sintomi}: RMS $>$ 2" per osservazioni moderne.

\textbf{Cause}:
\begin{itemize}
    \item Errori sistematici nelle osservazioni
    \item Coordinate osservatorio errate
    \item Errori di temporizzazione
    \item Perturbazioni mancanti (es., incontro ravvicinato)
\end{itemize}

\textbf{Diagnosi}: Graficare residui vs. tempo, magnitudine, osservatorio.

\subsection{Residui Piccoli ma Orbita Sbagliata}

\textbf{Sintomi}: RMS $<$ 0.5" ma predizioni effemeridi falliscono.

\textbf{Causa}: Arco corto + degenerazione. Molte orbite si adattano ugualmente bene su archi brevi.

\textbf{Soluzione}: Acquisire osservazioni su arco più lungo (>30 giorni per fascia principale, >7 giorni per NEA).

\section{Sommario}

Punti chiave sulla correzione differenziale:

\begin{enumerate}
    \item I \textbf{minimi quadrati} minimizzano la somma pesata dei residui O-C al quadrato
    \item Le \textbf{equazioni normali} $\mathbf{N}\Delta\mathbf{y}_0 = \mathbf{b}$ vengono risolte iterativamente
    \item Le \textbf{derivate parziali} vengono calcolate via regola della catena con STM
    \item Le \textbf{derivate geometriche} relazionano RA/Dec alla posizione topocentrica
    \item \textbf{Convergenza} tipicamente in 3-10 iterazioni
    \item La \textbf{matrice di covarianza} $\mathbf{C} = \mathbf{N}^{-1}$ fornisce l'incertezza formale
    \item L'\textbf{RMS} indica qualità del fit; obiettivo $<$1" per CCD moderno
    \item La \textbf{pesatura robusta} riduce il peso degli outlier
    \item L'\textbf{esempio Pompeja} dimostra il workflow completo
\end{enumerate}

Il prossimo capitolo copre l'analisi dei residui per valutazione qualità e rilevamento outlier.

\include{15_residui}

% Parte IV: Implementazione Libreria
\part{Implementazione della Libreria AstDyn}
\chapter{Architettura Software}
\label{ch:architecture}

\section{Introduzione}

AstDyn è progettata come una moderna libreria C++17 per astrodinamica e determinazione orbitale. L'architettura enfatizza:

\begin{itemize}
    \item \textbf{Modularità}: Moduli indipendenti con interfacce chiare
    \item \textbf{Prestazioni}: Algoritmi numerici efficienti con Eigen3
    \item \textbf{Estensibilità}: Facile aggiungere nuovi modelli di forza, integratori, parser
    \item \textbf{Manutenibilità}: Codice pulito, test completi, documentazione
\end{itemize}

\section{Principi di Design}

\subsection{Separazione delle Responsabilità}

Ogni modulo gestisce un aspetto specifico:
\begin{itemize}
    \item \textbf{Time}: Conversioni tra scale temporali (UTC, TT, TDB)
    \item \textbf{Coordinates}: Sistemi di riferimento, trasformazioni
    \item \textbf{Orbit}: Elementi, vettori di stato, conversioni
    \item \textbf{Propagation}: Integrazione numerica, modelli di forza
    \item \textbf{Observations}: Astrometria, formato MPC, pesi
    \item \textbf{Orbit Determination}: IOD, correzione differenziale, residui
\end{itemize}

\subsection{Design Basato su Interfacce}

Interfacce astratte abilitano flessibilità:

\begin{lstlisting}[language=C++,caption={Esempi di interfacce}]
// Interfaccia parser - supporto formati multipli
class IParser {
public:
    virtual ~IParser() = default;
    virtual OrbitalElements parse(const std::string& filename) = 0;
};

// Interfaccia integratore - metodi multipli disponibili
class IIntegrator {
public:
    virtual ~IIntegrator() = default;
    virtual void integrate(State& y, double t0, double t1, ForceModel& forces) = 0;
};

// Interfaccia effemeridi - SPICE, JPL, analitico
class IEphemeris {
public:
    virtual ~IEphemeris() = default;
    virtual Vector3d get_position(Body body, double jd_tdb) = 0;
};
\end{lstlisting}

\subsection{Header-Only vs. Compilato}

\textbf{Header-only} (inline, template):
\begin{itemize}
    \item \texttt{core/Constants.hpp}: Costanti fisiche
    \item \texttt{core/Types.hpp}: Alias di tipo, enum
    \item \texttt{utils/StringUtils.hpp}: Utilità per stringhe
\end{itemize}

\textbf{Compilato} (implementazione in .cpp):
\begin{itemize}
    \item Tutti gli algoritmi numerici (propagazione, integrazione)
    \item Operazioni I/O (parsing file, caricamento osservazioni)
    \item Calcoli complessi (STM, correzione differenziale)
\end{itemize}

\section{Organizzazione dei Moduli}

\subsection{Struttura delle Directory}

\begin{lstlisting}[language=bash,caption={Layout del progetto}]
astdyn/
|-- include/astdyn/           # Header pubblici
|   |-- AstDyn.hpp           # Include principale (tutto)
|   |-- AstDynEngine.hpp     # Engine alto livello
|   |-- Version.hpp          # Info versione (generato)
|   |-- Config.hpp           # Configurazione build (generato)
|   |-- core/                # Tipi fondamentali
|   |   |-- Constants.hpp
|   |   `-- Types.hpp
|   |-- math/                # Utilita' matematiche
|   |   |-- MathUtils.hpp
|   |   `-- LinearAlgebra.hpp
|   |-- time/                # Scale temporali
|   |   `-- TimeScale.hpp
|   |-- coordinates/         # Sistemi di riferimento
|   |   |-- KeplerianElements.hpp
|   |   |-- CartesianState.hpp
|   |   `-- CometaryElements.hpp
|   |-- orbit/               # Meccanica orbitale
|   |   |-- TwoBody.hpp
|   |   `-- Perturbations.hpp
|   |-- propagation/         # Integrazione numerica
|   |   |-- Integrator.hpp
|   |   `-- Propagator.hpp
|   |-- observations/        # Dati astrometrici
|   |   |-- Observation.hpp
|   |   |-- MPCReader.hpp
|   |   `-- ObservatoryDatabase.hpp
|   |-- orbit_determination/ # Algoritmi OD
|   |   |-- GaussIOD.hpp
|   |   |-- DifferentialCorrection.hpp
|   |   |-- StateTransitionMatrix.hpp
|   |   `-- Residuals.hpp
|   |-- io/                  # Parser
|   |   |-- IParser.hpp
|   |   |-- ParserFactory.hpp
|   |   `-- parsers/
|   |       |-- OrbFitEQ1Parser.hpp
|   |       `-- OrbFitRWOParser.hpp
|   |-- ephemeris/           # Posizioni planetarie
|   |   `-- SpiceInterface.hpp
|   `-- utils/               # Utilita'
|       |-- Logger.hpp
|       `-- StringUtils.hpp
|-- src/                     # File di implementazione
|   |-- CMakeLists.txt
|   |-- AstDynEngine.cpp
|   |-- math/
|   |-- time/
|   |-- coordinates/
|   |-- orbit/
|   |-- propagation/
|   |-- observations/
|   |-- orbit_determination/
|   |-- io/
|   `-- ephemeris/
|-- tests/                   # Test unitari (Google Test)
|-- examples/                # Programmi di esempio
|-- docs/                    # Documentazione
`-- data/                    # File dati (kernel, cataloghi)
\end{lstlisting}

\subsection{Organizzazione dei Namespace}

\begin{lstlisting}[language=C++,caption={Gerarchia namespace}]
namespace astdyn {
    namespace constants {    // Costanti fisiche
        constexpr double AU = 149597870.7;  // km
        constexpr double C_LIGHT = 299792.458;  // km/s
        // ...
    }
    
    namespace math {         // Utilita' matematiche
        double mod_angle(double angle, double period);
        Matrix3d rotation_matrix_z(double angle);
        // ...
    }
    
    namespace time {         // Conversioni temporali
        double utc_to_tt(double jd_utc);
        double tt_to_tdb(double jd_tt);
        // ...
    }
    
    namespace coordinates {  // Sistemi di coordinate
        class KeplerianElements { /* ... */ };
        class CartesianState { /* ... */ };
        // ...
    }
    
    namespace observations { // Osservazioni
        class Observation { /* ... */ };
        class MPCReader { /* ... */ };
        // ...
    }
    
    // Propagazione, determinazione orbitale a livello superiore
    class Propagator { /* ... */ };
    class DifferentialCorrection { /* ... */ };
    // ...
}
\end{lstlisting}

\section{Componenti Core}

\subsection{Costanti e Tipi}

\textbf{Costanti Fisiche} (\texttt{core/Constants.hpp}):
\begin{itemize}
    \item Parametri gravitazionali: \texttt{MU\_SUN}, \texttt{MU\_EARTH}, ecc.
    \item Distanze: \texttt{AU}, \texttt{EARTH\_RADIUS}
    \item Tempo: \texttt{JD2000}, \texttt{SECONDS\_PER\_DAY}
    \item Velocità della luce, obliquità, ecc.
\end{itemize}

\textbf{Alias di Tipo} (\texttt{core/Types.hpp}):
\begin{lstlisting}[language=C++]
// Algebra lineare (Eigen)
using Vector3d = Eigen::Vector3d;
using Vector6d = Eigen::Matrix<double, 6, 1>;
using Matrix3d = Eigen::Matrix3d;
using Matrix6d = Eigen::Matrix<double, 6, 6>;

// Tipizzazione forte per unita'
using Radians = double;
using Degrees = double;
using AU_Distance = double;
using KM_Distance = double;
using JulianDate = double;
\end{lstlisting}

\textbf{Enumerazioni}:
\begin{lstlisting}[language=C++]
enum class CoordinateSystem {
    ECLIPTIC, EQUATORIAL, ICRF, BODY_FIXED
};

enum class ElementType {
    KEPLERIAN, CARTESIAN, COMETARY, EQUINOCTIAL
};

enum class TimeScale {
    UTC, UT1, TAI, TT, TDB, TCB, TCG
};

enum class IntegratorType {
    RADAU15, RK_GAUSS, DOPRI, LSODAR, GAUSS_JACKSON
};
\end{lstlisting}

\subsection{Versione e Configurazione}

\textbf{Versione} (generata da CMake):
\begin{lstlisting}[language=C++]
namespace astdyn {
    namespace Version {
        constexpr int major = 1;
        constexpr int minor = 0;
        constexpr int patch = 0;
        constexpr const char* string = "1.0.0";
    }
}
\end{lstlisting}

\textbf{Configurazione} (opzioni di build):
\begin{lstlisting}[language=C++]
namespace astdyn {
    namespace Config {
        constexpr bool use_spice = true;
        constexpr bool use_openmp = false;
        constexpr const char* build_type = "Release";
        constexpr const char* compiler = "AppleClang 16.0.0";
    }
}
\end{lstlisting}

\section{Gestione delle Dipendenze}

\subsection{Dipendenze Esterne}

\textbf{Eigen3} (richiesto):
\begin{itemize}
    \item Scopo: Algebra lineare (vettori, matrici)
    \item Versione: $\ge$ 3.3
    \item Uso: Header-only, nessun linking richiesto
    \item Motivazione: Veloce, espressivo, basato su template
\end{itemize}

\textbf{Boost} (opzionale):
\begin{itemize}
    \item Scopo: Utilità estese (filesystem, date\_time)
    \item Versione: $\ge$ 1.70
    \item Uso: Alcuni componenti compilati
    \item Motivazione: Estensioni C++ standard industriale
\end{itemize}

\textbf{SPICE} (opzionale):
\begin{itemize}
    \item Scopo: Effemeridi planetarie ad alta precisione
    \item Provider: JPL/NAIF
    \item Uso: Libreria compilata (CSPICE)
    \item Motivazione: Standard di riferimento per calcolo effemeridi
\end{itemize}

\textbf{Google Test} (solo testing):
\begin{itemize}
    \item Scopo: Framework per test unitari
    \item Versione: $\ge$ 1.10
    \item Uso: Scaricato automaticamente da CMake se non trovato
\end{itemize}

\subsection{Sistema di Build CMake}

\textbf{Funzionalità}:
\begin{itemize}
    \item CMake moderno (3.15+)
    \item Ricerca automatica dipendenze
    \item Generazione versione
    \item Opzioni di configurazione
    \item Target di installazione
    \item Export pacchetto per uso in altri progetti
\end{itemize}

\textbf{Opzioni di build}:
\begin{lstlisting}[language=bash]
cmake -B build \
  -DCMAKE_BUILD_TYPE=Release \
  -DASTDYN_BUILD_SHARED=ON \
  -DASTDYN_BUILD_TESTS=ON \
  -DASTDYN_BUILD_EXAMPLES=ON \
  -DASTDYN_USE_SPICE=ON
cmake --build build -j
cmake --install build
\end{lstlisting}

\section{Gestione degli Errori}

\subsection{Strategia}

\textbf{Eccezioni} per errori di programmazione:
\begin{lstlisting}[language=C++]
if (eccentricity < 0.0 || eccentricity >= 1.0) {
    throw std::invalid_argument("L'eccentricita' deve essere in [0, 1)");
}
\end{lstlisting}

\textbf{Optional} per fallimenti attesi:
\begin{lstlisting}[language=C++]
std::optional<Matrix3d> invert_matrix(const Matrix3d& A) {
    if (A.determinant() < 1e-15) {
        return std::nullopt;  // Singolare
    }
    return A.inverse();
}
\end{lstlisting}

\textbf{Codici di ritorno} per I/O:
\begin{lstlisting}[language=C++]
bool load_observations(const std::string& filename,
                       std::vector<Observation>& obs) {
    std::ifstream file(filename);
    if (!file) return false;
    // ...
    return true;
}
\end{lstlisting}

\subsection{Logging}

\begin{lstlisting}[language=C++]
#include <astdyn/utils/Logger.hpp>

// Livelli di gravita'
Logger::debug("Iterazione {} convergenza", iter);
Logger::info("Caricate {} osservazioni", n_obs);
Logger::warning("RMS = {:.3f} arcosecondi (alto!)", rms);
Logger::error("Caricamento kernel fallito: {}", filename);
\end{lstlisting}

\section{Gestione della Memoria}

\subsection{Ownership}

\textbf{Allocazione stack} per oggetti piccoli:
\begin{lstlisting}[language=C++]
Vector3d position;  // 24 byte
Matrix6d covariance;  // 288 byte
KeplerianElements elements;  // ~80 byte
\end{lstlisting}

\textbf{Smart pointer} per durata dinamica:
\begin{lstlisting}[language=C++]
// Ownership unico
auto propagator = std::make_unique<Propagator>(integrator, forces);

// Ownership condiviso (quando servono riferimenti multipli)
auto spice = std::make_shared<SpiceInterface>();
propagator->set_ephemeris(spice);
corrector->set_ephemeris(spice);  // Stesso oggetto
\end{lstlisting}

\textbf{Move semantics} per efficienza:
\begin{lstlisting}[language=C++]
std::vector<Observation> load_mpc_observations(const std::string& file) {
    std::vector<Observation> obs;
    // ... popola obs ...
    return obs;  // Spostato, non copiato (RVO C++11)
}
\end{lstlisting}

\subsection{Dataset Grandi}

Per grandi set di osservazioni (es. 10.000+ osservazioni):
\begin{itemize}
    \item Usare \texttt{std::vector::reserve()} per evitare riallocazioni
    \item Elaborazione stream per file troppo grandi per RAM
    \item File memory-mapped per dataset molto grandi (futuro)
\end{itemize}

\section{Threading e Parallelismo}

\subsection{Stato Attuale}

AstDyn v1.0 è single-thread. Opportunità di parallelizzazione:

\begin{enumerate}
    \item \textbf{Elaborazione osservazioni}: Calcolo partiali in parallelo
    \item \textbf{Monte Carlo}: Propagazioni orbitali multiple indipendenti
    \item \textbf{Propagazione incertezza}: Simulazioni particelle parallele
\end{enumerate}

\subsection{Piani Futuri}

\begin{lstlisting}[language=C++]
// OpenMP per parallelizzazione loop
#pragma omp parallel for
for (size_t i = 0; i < observations.size(); ++i) {
    residuals[i] = compute_residual(observations[i], state);
}

// std::async per parallelismo task
auto future1 = std::async(std::launch::async, propagate, state1, t_end);
auto future2 = std::async(std::launch::async, propagate, state2, t_end);
auto result1 = future1.get();
auto result2 = future2.get();
\end{lstlisting}

\section{Strategia di Testing}

\subsection{Test Unitari}

Framework Google Test con fixture:

\begin{lstlisting}[language=C++]
TEST(TimeScaleTest, UTCtoTT) {
    double jd_utc = 2451545.0;  // J2000.0
    double jd_tt = time::utc_to_tt(jd_utc);
    EXPECT_NEAR(jd_tt - jd_utc, 64.184 / 86400.0, 1e-10);
}

TEST(KeplerianTest, CartesianRoundTrip) {
    CartesianState cart(1.0, 0.0, 0.0, 0.0, 0.0172, 0.0);
    auto kep = KeplerianElements::from_cartesian(cart);
    auto cart2 = kep.to_cartesian();
    EXPECT_VECTOR_NEAR(cart.position, cart2.position, 1e-12);
}
\end{lstlisting}

\subsection{Test di Integrazione}

\begin{itemize}
    \item Propagazione orbite note, confronto con JPL Horizons
    \item Correzione differenziale su asteroidi reali (es. Pompeja)
    \item IOD da osservazioni sintetiche
\end{itemize}

\subsection{Benchmark Prestazionali}

\begin{lstlisting}[language=C++]
TEST(PropagationBenchmark, Pompeja60Days) {
    auto start = std::chrono::high_resolution_clock::now();
    
    propagate(initial_state, 0.0, 60.0, forces);
    
    auto end = std::chrono::high_resolution_clock::now();
    auto duration = std::chrono::duration_cast<std::chrono::milliseconds>(end - start);
    
    std::cout << "Tempo propagazione: " << duration.count() << " ms\n";
    EXPECT_LT(duration.count(), 1000);  // Deve completare in < 1 secondo
}
\end{lstlisting}

\section{Documentazione}

\subsection{Documentazione Inline}

Commenti stile Doxygen:

\begin{lstlisting}[language=C++]
/**
 * @brief Converte elementi kepleriani in stato cartesiano
 * 
 * @param elements Elementi orbitali kepleriani (a, e, i, Omega, omega, M)
 * @param mu Parametro gravitazionale [km^3/s^2]
 * @return CartesianState Posizione [km] e velocita' [km/s]
 * 
 * @note Usa soluzione iterativa equazione di Keplero per anomalia eccentrica
 * @throws std::invalid_argument se eccentricita' >= 1.0 (parabolica/iperbolica)
 */
CartesianState to_cartesian(const KeplerianElements& elements, double mu);
\end{lstlisting}

\subsection{Documentazione Esterna}

\begin{itemize}
    \item \textbf{README.md}: Avvio rapido, installazione, esempi
    \item \textbf{Questo manuale}: Teoria + implementazione
    \item \textbf{Riferimento API}: Generato da Doxygen
    \item \textbf{Esempi}: Codice funzionante commentato
\end{itemize}

\section{Riepilogo}

Caratteristiche architetturali chiave:

\begin{enumerate}
    \item \textbf{Design modulare}: Chiara separazione delle responsabilità
    \item \textbf{Basato su interfacce}: Facile estendere (parser, integratori, ecc.)
    \item \textbf{C++17 moderno}: Smart pointer, move semantics, template
    \item \textbf{Integrazione Eigen3}: Algebra lineare efficiente
    \item \textbf{Build CMake}: Multi-piattaforma, dipendenze automatiche
    \item \textbf{Testing completo}: Test unitari + test integrazione
    \item \textbf{Gestione errori chiara}: Eccezioni, optional, codici ritorno
    \item \textbf{Ben documentato}: Documentazione inline + esterna
\end{enumerate}

Il prossimo capitolo copre nel dettaglio i singoli moduli core.

\chapter{Moduli Core}
\label{ch:core_modules}

\section{Introduzione}

Questo capitolo documenta i moduli core che implementano gli algoritmi di meccanica orbitale. Ogni modulo è progettato per essere indipendente ma componibile.

\section{Elementi Orbitali}

\subsection{KeplerianElements}

Sei elementi kepleriani classici per orbite ellittiche.

\begin{lstlisting}[language=C++,caption={Classe KeplerianElements}]
namespace astdyn {
namespace coordinates {

class KeplerianElements {
public:
    // Elementi
    double a;      // Semiasse maggiore [AU]
    double e;      // Eccentricita' [0, 1)
    double i;      // Inclinazione [rad]
    double Omega;  // Longitudine nodo ascendente [rad]
    double omega;  // Argomento perielio [rad]
    double M;      // Anomalia media [rad]
    
    // Epoca
    double epoch;  // Data giuliana [TDB]
    
    // Costruzione
    KeplerianElements() = default;
    KeplerianElements(double a, double e, double i,
                     double Omega, double omega, double M,
                     double epoch);
    
    // Conversioni
    static KeplerianElements from_cartesian(
        const Vector6d& state, double epoch, double mu = MU_SUN);
    
    Vector6d to_cartesian(double mu = MU_SUN) const;
    
    // Quantita' derivate
    double period() const;           // Periodo orbitale [giorni]
    double mean_motion() const;      // Moto medio [rad/giorno]
    double perihelion_distance() const;  // q [AU]
    double aphelion_distance() const;    // Q [AU]
    double orbital_energy(double mu = MU_SUN) const;
    
    // Anomalia media a epoca diversa
    double mean_anomaly_at(double jd) const;
    
    // Validazione
    bool is_valid() const;
};

}} // namespace
\end{lstlisting}

\textbf{Uso}:
\begin{lstlisting}[language=C++]
using namespace astdyn::coordinates;

// Crea da elementi
KeplerianElements elem;
elem.a = 2.77;          // AU
elem.e = 0.075;
elem.i = 10.6 * DEG_TO_RAD;
elem.Omega = 80.3 * DEG_TO_RAD;
elem.omega = 73.6 * DEG_TO_RAD;
elem.M = 0.0;
elem.epoch = 2460000.5;

// Quantita' derivate
std::cout << "Periodo: " << elem.period() << " giorni\n";
std::cout << "q: " << elem.perihelion_distance() << " AU\n";

// Converti in cartesiano
Vector6d state = elem.to_cartesian();
\end{lstlisting}

\subsection{CometaryElements}

Ottimizzato per orbite paraboliche e quasi-paraboliche (comete).

\begin{lstlisting}[language=C++]
class CometaryElements {
public:
    double q;      // Distanza perielio [AU]
    double e;      // Eccentricita'
    double i;      // Inclinazione [rad]
    double Omega;  // Longitudine nodo ascendente [rad]
    double omega;  // Argomento perielio [rad]
    double T;      // Tempo passaggio perielio [JD]
    double epoch;
    
    Vector6d to_cartesian(double jd, double mu = MU_SUN) const;
    static CometaryElements from_keplerian(const KeplerianElements& kep);
};
\end{lstlisting}

\subsection{CartesianState}

Vettori posizione e velocità.

\begin{lstlisting}[language=C++]
struct CartesianState {
    Vector3d position;  // [AU]
    Vector3d velocity;  // [AU/giorno]
    double epoch;       // [JD TDB]
    
    Vector6d as_vector() const {
        Vector6d v;
        v << position, velocity;
        return v;
    }
    
    double distance() const { return position.norm(); }
    double speed() const { return velocity.norm(); }
};
\end{lstlisting}

\section{Modelli di Forza}

\subsection{Interfaccia ForceModel}

\begin{lstlisting}[language=C++]
class ForceModel {
public:
    virtual ~ForceModel() = default;
    
    // Calcola accelerazione [AU/giorno^2]
    virtual Vector3d acceleration(
        const Vector6d& state,
        double jd_tdb) const = 0;
    
    // Derivate parziali per STM (opzionale)
    virtual Matrix3d acceleration_partials_position(
        const Vector6d& state,
        double jd_tdb) const {
        return Matrix3d::Zero();
    }
    
    virtual Matrix3d acceleration_partials_velocity(
        const Vector6d& state,
        double jd_tdb) const {
        return Matrix3d::Zero();
    }
};
\end{lstlisting}

\subsection{Gravità a Massa Puntiforme}

\begin{lstlisting}[language=C++]
class PointMassGravity : public ForceModel {
private:
    std::shared_ptr<IEphemeris> ephemeris_;
    std::vector<Body> bodies_;  // Sole, pianeti
    
public:
    PointMassGravity(std::shared_ptr<IEphemeris> eph,
                     const std::vector<Body>& bodies)
        : ephemeris_(eph), bodies_(bodies) {}
    
    Vector3d acceleration(const Vector6d& state, double jd) const override {
        Vector3d r_obj = state.head<3>();
        Vector3d acc = Vector3d::Zero();
        
        for (Body body : bodies_) {
            Vector3d r_body = ephemeris_->get_position(body, jd);
            Vector3d d = r_body - r_obj;
            double d_norm = d.norm();
            
            // Termine diretto
            acc += body.mu * d / (d_norm * d_norm * d_norm);
            
            // Termine indiretto (se non e' il Sole)
            if (body != Body::SUN) {
                double r_norm = r_body.norm();
                acc -= body.mu * r_body / (r_norm * r_norm * r_norm);
            }
        }
        
        return acc;
    }
};
\end{lstlisting}

\subsection{Modello di Forza Combinato}

\begin{lstlisting}[language=C++]
class CombinedForceModel : public ForceModel {
private:
    std::vector<std::shared_ptr<ForceModel>> models_;
    
public:
    void add_model(std::shared_ptr<ForceModel> model) {
        models_.push_back(model);
    }
    
    Vector3d acceleration(const Vector6d& state, double jd) const override {
        Vector3d acc = Vector3d::Zero();
        for (const auto& model : models_) {
            acc += model->acceleration(state, jd);
        }
        return acc;
    }
};
\end{lstlisting}

\section{Integrazione Numerica}

\subsection{Interfaccia Integrator}

\begin{lstlisting}[language=C++]
class IIntegrator {
public:
    virtual ~IIntegrator() = default;
    
    // Passo singolo
    virtual void step(Vector6d& y, double& t, double dt,
                     const ForceModel& forces) = 0;
    
    // Integra da t0 a t1
    virtual void integrate(Vector6d& y, double t0, double t1,
                          const ForceModel& forces,
                          double dt_initial = 0.01) = 0;
    
    // Ottieni statistiche
    virtual size_t num_steps() const = 0;
    virtual size_t num_function_calls() const = 0;
};
\end{lstlisting}

\subsection{Runge-Kutta-Fehlberg 7(8)}

Dimensione passo adattiva, alta accuratezza.

\begin{lstlisting}[language=C++]
class RKF78 : public IIntegrator {
private:
    double tol_;         // Tolleranza errore
    double dt_min_;      // Dimensione passo minima
    double dt_max_;      // Dimensione passo massima
    size_t n_steps_;
    size_t n_fcalls_;
    
public:
    RKF78(double tol = 1e-12,
          double dt_min = 1e-6,
          double dt_max = 100.0)
        : tol_(tol), dt_min_(dt_min), dt_max_(dt_max),
          n_steps_(0), n_fcalls_(0) {}
    
    void integrate(Vector6d& y, double t0, double t1,
                  const ForceModel& forces,
                  double dt) override {
        double t = t0;
        double h = dt;
        
        while (t < t1) {
            if (t + h > t1) h = t1 - t;
            
            // Coefficienti e stadi RKF78 (13 stadi)
            Vector6d k[13];
            // ... calcola stadi ...
            
            // Soluzioni 7° e 8° ordine
            Vector6d y7 = y + h * (/* combinazione 7° ordine */);
            Vector6d y8 = y + h * (/* combinazione 8° ordine */);
            
            // Stima errore
            double err = (y8 - y7).norm();
            
            // Accetta/rifiuta e adatta passo
            if (err < tol_) {
                y = y8;
                t += h;
                n_steps_++;
            }
            
            // Aggiorna dimensione passo
            h *= 0.9 * std::pow(tol_ / err, 1.0/8.0);
            h = std::clamp(h, dt_min_, dt_max_);
            
            n_fcalls_ += 13;
        }
    }
};
\end{lstlisting}

\section{Propagazione Orbitale}

\subsection{Classe Propagator}

Interfaccia alto livello che combina integratore e forze.

\begin{lstlisting}[language=C++]
class Propagator {
private:
    std::shared_ptr<IIntegrator> integrator_;
    std::shared_ptr<ForceModel> forces_;
    std::shared_ptr<IEphemeris> ephemeris_;
    
public:
    Propagator(std::shared_ptr<IIntegrator> integ,
               std::shared_ptr<ForceModel> forces,
               std::shared_ptr<IEphemeris> eph)
        : integrator_(integ), forces_(forces), ephemeris_(eph) {}
    
    // Propaga stato
    Vector6d propagate(const Vector6d& y0, double t0, double t1) {
        Vector6d y = y0;
        integrator_->integrate(y, t0, t1, *forces_);
        return y;
    }
    
    // Propaga con STM
    std::pair<Vector6d, Matrix6d> propagate_with_stm(
        const Vector6d& y0, double t0, double t1) {
        
        // Stato aumentato: [y, Phi(vettorizzata)]
        VectorXd aug(42);  // 6 + 36
        aug.head<6>() = y0;
        aug.tail<36>() = Matrix6d::Identity().reshaped();
        
        // Integra equazioni variazionali
        integrator_->integrate(aug, t0, t1, *forces_);
        
        Vector6d y = aug.head<6>();
        Matrix6d Phi = Map<Matrix6d>(aug.tail<36>().data());
        
        return {y, Phi};
    }
    
    // Genera tabella effemeridi
    std::vector<std::pair<double, Vector6d>> 
    generate_ephemeris(const Vector6d& y0, double t0,
                      double t1, double dt) {
        std::vector<std::pair<double, Vector6d>> table;
        Vector6d y = y0;
        double t = t0;
        
        while (t <= t1) {
            table.emplace_back(t, y);
            if (t + dt > t1) dt = t1 - t;
            integrator_->integrate(y, t, t + dt, *forces_);
            t += dt;
        }
        
        return table;
    }
};
\end{lstlisting}

\textbf{Esempio d'uso}:
\begin{lstlisting}[language=C++]
// Configurazione
auto spice = std::make_shared<SpiceInterface>();
spice->load_kernel("de440.bsp");

auto forces = std::make_shared<PointMassGravity>(
    spice, {Body::SUN, Body::JUPITER, Body::SATURN});

auto integrator = std::make_shared<RKF78>(1e-12);

Propagator prop(integrator, forces, spice);

// Propaga Pompeja per 60 giorni
Vector6d y0 = /* stato iniziale */;
double t0 = 2460000.5;
double t1 = t0 + 60.0;

Vector6d y_final = prop.propagate(y0, t0, t1);

std::cout << "Posizione finale: " << y_final.head<3>().transpose() << " AU\n";
\end{lstlisting}

\section{Osservazioni}

\subsection{Classe Observation}

\begin{lstlisting}[language=C++]
namespace astdyn {
namespace observations {

struct Observation {
    double epoch;        // JD UTC
    double ra;           // Ascensione retta [rad]
    double dec;          // Declinazione [rad]
    double sigma_ra;     // Incertezza RA [rad]
    double sigma_dec;    // Incertezza Dec [rad]
    std::string obs_code; // Codice osservatorio MPC
    double magnitude;    // Magnitudine apparente
    
    // Calcolata da RA/Dec
    Vector3d line_of_sight() const {
        return Vector3d(
            std::cos(dec) * std::cos(ra),
            std::cos(dec) * std::sin(ra),
            std::sin(dec)
        );
    }
    
    // Peso per minimi quadrati
    double weight_ra() const { return 1.0 / (sigma_ra * sigma_ra); }
    double weight_dec() const { return 1.0 / (sigma_dec * sigma_dec); }
};

}} // namespace
\end{lstlisting}

\subsection{MPC Reader}

Parsing formato 80 colonne del Minor Planet Center.

\begin{lstlisting}[language=C++]
class MPCReader {
public:
    static std::vector<Observation> read_file(const std::string& filename) {
        std::vector<Observation> obs;
        std::ifstream file(filename);
        std::string line;
        
        while (std::getline(file, line)) {
            if (line.length() < 80) continue;
            if (line[14] == 'S' || line[14] == 'X') continue; // Satellite/roving
            
            Observation ob;
            
            // Parsing colonne (specifica formato MPC)
            ob.obs_code = line.substr(77, 3);
            
            // Data/ora
            int year = std::stoi(line.substr(15, 4));
            int month = std::stoi(line.substr(20, 2));
            double day = std::stod(line.substr(23, 8));
            ob.epoch = date_to_jd(year, month, day);
            
            // RA: HH MM SS.sss
            int ra_h = std::stoi(line.substr(32, 2));
            int ra_m = std::stoi(line.substr(35, 2));
            double ra_s = std::stod(line.substr(38, 5));
            ob.ra = (ra_h + ra_m/60.0 + ra_s/3600.0) * 15.0 * DEG_TO_RAD;
            
            // Dec: +DD MM SS.ss
            char sign = line[44];
            int dec_d = std::stoi(line.substr(45, 2));
            int dec_m = std::stoi(line.substr(48, 2));
            double dec_s = std::stod(line.substr(51, 4));
            ob.dec = (dec_d + dec_m/60.0 + dec_s/3600.0) * DEG_TO_RAD;
            if (sign == '-') ob.dec = -ob.dec;
            
            // Magnitudine
            if (line.length() >= 70 && line[65] != ' ') {
                ob.magnitude = std::stod(line.substr(65, 5));
            }
            
            // Incertezze predefinite (dipendenti da catalogo)
            ob.sigma_ra = 0.5 * ARCSEC_TO_RAD;
            ob.sigma_dec = 0.5 * ARCSEC_TO_RAD;
            
            obs.push_back(ob);
        }
        
        return obs;
    }
};
\end{lstlisting}

\section{Database Osservatori}

\subsection{ObservatoryCoordinates}

\begin{lstlisting}[language=C++]
struct ObservatoryCoordinates {
    std::string code;
    double longitude;  // [rad] Est positivo
    double latitude;   // [rad] geocentrico
    double altitude;   // [m] sopra livello mare
    
    // Posizione geocentrica a tempo dato
    Vector3d position_itrf(double jd_utc) const {
        // Ellissoide WGS84
        const double a = 6378137.0;  // m
        const double f = 1.0 / 298.257223563;
        const double e2 = 2*f - f*f;
        
        double N = a / std::sqrt(1 - e2 * std::sin(latitude) * std::sin(latitude));
        
        double x = (N + altitude) * std::cos(latitude) * std::cos(longitude);
        double y = (N + altitude) * std::cos(latitude) * std::sin(longitude);
        double z = (N * (1 - e2) + altitude) * std::sin(latitude);
        
        return Vector3d(x, y, z) / 1000.0;  // Converti in km
    }
    
    // Ruota a sistema inerziale
    Vector3d position_icrf(double jd_utc) const {
        Vector3d r_itrf = position_itrf(jd_utc);
        Matrix3d R = earth_rotation_matrix(jd_utc);  // ITRF -> ICRF
        return R * r_itrf / AU;  // Converti in AU
    }
};
\end{lstlisting}

\section{Riepilogo}

I moduli core forniscono:

\begin{enumerate}
    \item \textbf{Elementi Orbitali}: Rappresentazioni kepleriane, cartesiane, cometarie
    \item \textbf{Modelli di Forza}: Interfaccia estensibile per perturbazioni
    \item \textbf{Integratori}: Metodi RK con dimensione passo adattiva
    \item \textbf{Propagator}: Propagazione orbitale alto livello con STM
    \item \textbf{Osservazioni}: Misure astrometriche e parsing MPC
    \item \textbf{Osservatori}: Coordinate geodetiche e trasformazioni
\end{enumerate}

Tutti i moduli sono progettati per composizione ed estensibilità.

\include{18_parser}
\include{19_riferimento_api}
\include{20_esempi}

% Parte V: Validazione e Applicazioni
\part{Validazione e Applicazioni}
\include{21_validazione}
\include{22_casi_studio}
\include{23_prestazioni}

% Parte VI: Prospettive Future
\part{Prospettive Future}
\chapter{Sviluppi Futuri}
\label{ch:future}

\section{Introduzione}

Questo capitolo delinea i miglioramenti pianificati e le direzioni di ricerca future per AstDyn. Sebbene l'attuale release v1.0 fornisca capacità di determinazione orbitale di livello produttivo, diverse estensioni potrebbero ampliare funzionalità e prestazioni.

\section{Forze Non Gravitazionali}

\subsection{Pressione di Radiazione Solare}

\textbf{Stato}: Implementazione parziale esistente (esempio nel Capitolo 20).

\textbf{Pianificato}:
\begin{itemize}
    \item Integrazione completa nel framework dei modelli di forza
    \item Modello di ombreggiamento (occultazione Terra/Luna)
    \item Ri-irradiazione termica (effetto Yarkovsky)
    \item Stima dei parametri nella correzione differenziale
\end{itemize}

\textbf{Schema di implementazione}:
\begin{lstlisting}[language=C++]
class SolarRadiationPressure : public ForceModel {
public:
    SolarRadiationPressure(
        double area_mass_ratio,
        double reflectivity = 1.0,
        bool include_yarkovsky = false
    );
    
    Vector3d acceleration(double t, const Vector3d& pos, 
                         const Vector3d& vel) const override;
    
    // Per correzione differenziale
    bool supports_partials() const override { return true; }
    std::pair<Matrix3d, Matrix3d> partials(...) const override;
    
private:
    double area_mass_ratio_;
    double reflectivity_;
    bool include_yarkovsky_;
    
    // Calcolo dell'ombra
    double shadow_function(double t, const Vector3d& pos) const;
};
\end{lstlisting}

\textbf{Impatto scientifico}: Critico per piccoli NEA e tracciamento detriti spaziali.

\subsection{Degassamento Cometario}

\textbf{Motivazione}: Le comete esibiscono accelerazioni non gravitazionali dovute alla sublimazione di volatili.

\textbf{Modello pianificato}:
\begin{itemize}
    \item Formulazione di Marsden: termini $A_1/r^2 + A_2/r^3 + A_3$
    \item Componenti radiale, trasversale e normale
    \item Curva di attività dipendente dalla temperatura
\end{itemize}

\begin{lstlisting}[language=C++]
class CometaryOutgassing : public ForceModel {
public:
    CometaryOutgassing(double A1, double A2, double A3);
    
    Vector3d acceleration(double t, const Vector3d& pos, 
                         const Vector3d& vel) const override {
        Vector3d r_sun = pos;  // Posizione eliocentrica
        double r = r_sun.norm();
        
        // Modello di Marsden
        Vector3d radial = r_sun.normalized();
        Vector3d transverse = /* calcola dalla velocita' */;
        Vector3d normal = radial.cross(transverse);
        
        double g = activity_function(r);  // Curva di attivita'
        
        return g * (A1_ * radial + A2_ * transverse + A3_ * normal) / (r * r);
    }
    
private:
    double A1_, A2_, A3_;
    double activity_function(double r) const;
};
\end{lstlisting}

\textbf{Caso d'uso}: Comete a lungo periodo, comete a breve periodo con degassamento significativo.

\subsection{Relatività Generale}

\textbf{Limitazione attuale}: Effetti post-newtoniani trascurati.

\textbf{Pianificato}: Correzioni relativistiche del primo ordine.

\textbf{Formulazione}:
\[
\mathbf{a}_{\text{rel}} = \frac{GM_\odot}{c^2 r^3} \left[ 4 \frac{GM_\odot}{r} - v^2 \right] \mathbf{r} + 4 \frac{GM_\odot}{c^2 r^3} (\mathbf{r} \cdot \mathbf{v}) \mathbf{v}
\]

\textbf{Implementazione}:
\begin{lstlisting}[language=C++]
class RelativisticCorrection : public ForceModel {
public:
    Vector3d acceleration(double t, const Vector3d& pos, 
                         const Vector3d& vel) const override {
        double r = pos.norm();
        double v2 = vel.squaredNorm();
        double rdotv = pos.dot(vel);
        
        double factor1 = 4.0 * GM_SUN / r - v2;
        double factor2 = 4.0 * rdotv;
        
        return (GM_SUN / (C * C * r * r * r)) * 
               (factor1 * pos + factor2 * vel);
    }
};
\end{lstlisting}

\textbf{Magnitudine}: $\sim 10^{-8}$ m/s$^2$ a 1 AU—influenza significativamente Mercurio, trascurabile per asteroidi oltre Marte.

\section{Propagazione delle Incertezze}

\subsection{Propagazione della Covarianza}

\textbf{Attuale}: STM singola per iterazione.

\textbf{Pianificato}: Propagazione completa della matrice di covarianza con rumore di processo.

\begin{lstlisting}[language=C++]
class CovariancePropagator {
public:
    struct Result {
        CartesianState mean_state;
        Matrix6d covariance;
    };
    
    Result propagate_with_covariance(
        const CartesianState& initial_state,
        const Matrix6d& initial_covariance,
        double target_epoch,
        const Matrix6d& process_noise
    );
};
\end{lstlisting}

\textbf{Applicazione}: Previsione dell'incertezza, probabilità di collisione.

\subsection{Metodi Monte Carlo}

\textbf{Motivazione}: Propagazione non lineare dell'incertezza.

\textbf{Pianificato}:
\begin{lstlisting}[language=C++]
class MonteCarloUncertainty {
public:
    struct Sample {
        orbit::KeplerianElements elements;
        double weight;
    };
    
    std::vector<Sample> generate_samples(
        const orbit::KeplerianElements& nominal,
        const Matrix6d& covariance,
        size_t n_samples = 10000
    );
    
    std::vector<CartesianState> propagate_ensemble(
        const std::vector<Sample>& samples,
        double target_epoch
    );
    
    // Riepilogo statistico
    struct Statistics {
        CartesianState mean;
        CartesianState median;
        Matrix6d covariance;
        double position_rms;
    };
    
    Statistics compute_statistics(
        const std::vector<CartesianState>& ensemble
    );
};
\end{lstlisting}

\textbf{Caso d'uso}: Evitamento collisioni, analisi di avvicinamenti ravvicinati.

\section{Gestione degli Incontri Ravvicinati}

\subsection{Tecniche di Regolarizzazione}

\textbf{Problema}: Gli integratori standard hanno difficoltà con gli incontri planetari ravvicinati ($< 0.1$ AU).

\textbf{Pianificato}: Regolarizzazione di Kustaanheimo-Stiefel (KS) per rimozione delle singolarità.

\textbf{Trasformazione KS}:
\[
\mathbf{r} = \mathbf{u}^T L \mathbf{u}, \quad d\tau = r \, dt
\]

Trasforma il problema dei due corpi singolare in un oscillatore armonico regolare.

\textbf{Schema di implementazione}:
\begin{lstlisting}[language=C++]
class KSRegularizedIntegrator : public IIntegrator {
public:
    void integrate(double t0, double tf, std::vector<double>& y,
                  const std::function<...>& derivs) override {
        // Rileva avvicinamento ravvicinato
        if (is_close_approach(y)) {
            // Passa a coordinate KS
            auto u = cartesian_to_ks(y);
            // Integra nello spazio regolarizzato
            integrate_ks(t0, tf, u);
            // Trasforma indietro
            y = ks_to_cartesian(u);
        } else {
            // Integrazione standard
            standard_integrate(t0, tf, y, derivs);
        }
    }
};
\end{lstlisting}

\textbf{Beneficio}: Integrazione stabile attraverso sorvoli planetari.

\subsection{Analisi di Incontri Iperbolici}

\textbf{Funzionalità pianificate}:
\begin{itemize}
    \item Rilevamento automatico di incontri ravvicinati
    \item Parametri di targeting del piano B
    \item Calcolo della velocità e geometria dell'incontro
    \item Previsione degli elementi orbitali post-incontro
\end{itemize}

\section{Elaborazione Parallela}

\subsection{Parallelizzazione OpenMP}

\textbf{Limitazione attuale}: Solo single-thread.

\textbf{Operazioni target}:
\begin{enumerate}
    \item Propagazione batch di orbite
    \item Calcolo dei residui delle osservazioni
    \item Campionamento Monte Carlo
    \item Ricerca su griglia di parametri
\end{enumerate}

\textbf{Implementazione}:
\begin{lstlisting}[language=C++]
// Propagazione batch parallela
#pragma omp parallel for schedule(dynamic)
for (int i = 0; i < n_orbits; ++i) {
    auto state = propagator.propagate(initial_states[i], target_epoch);
    results[i] = state;
}

// Calcolo parallelo dei residui nella correzione differenziale
#pragma omp parallel for
for (int i = 0; i < n_observations; ++i) {
    auto computed = compute_predicted_observation(obs[i]);
    residuals[2*i] = obs[i].ra - computed.ra;
    residuals[2*i+1] = obs[i].dec - computed.dec;
}
\end{lstlisting}

\textbf{Speedup atteso}: 6-7$\times$ su CPU 8-core per operazioni batch.

\subsection{Accelerazione GPU}

\textbf{Obiettivo a lungo termine}: CUDA/OpenCL per parallelismo massiccio.

\textbf{Attività adatte}:
\begin{itemize}
    \item Incertezza Monte Carlo (10.000+ campioni)
    \item Generazione di tabelle di effemeridi
    \item Risoluzione batch di minimi quadrati
\end{itemize}

\textbf{Sfida}: Adattività della dimensione del passo di integrazione difficile su GPU.

\section{Integratori Aggiuntivi}

\subsection{Integratori Simplettici}

\textbf{Motivazione}: Conservazione dell'energia per studi di stabilità a lungo termine.

\textbf{Pianificato}: Mappa simplettica di Wisdom-Holman per sistemi gerarchici.

\begin{lstlisting}[language=C++]
class SymplecticIntegrator : public IIntegrator {
public:
    SymplecticIntegrator(double fixed_step_size);
    
    // Splitting dell'operatore: H = H_Kepler + H_interaction
    void step(double t, std::vector<double>& y, ...) override;
    
private:
    void drift_step(std::vector<double>& y, double dt);
    void kick_step(std::vector<double>& y, double dt);
};
\end{lstlisting}

\textbf{Caso d'uso}: Dinamica asteroidale su milioni di anni, stabilità di sistemi planetari.

\subsection{Metodi Impliciti}

\textbf{Pianificato}: Radau IIA per problemi rigidi (stiff).

\textbf{Vantaggio}: Stabilità incondizionata, buono per sistemi strettamente legati.

\textbf{Svantaggio}: Richiede calcolo dello Jacobiano, più lento per passo.

\section{Binding Python}

\subsection{Interfaccia pybind11}

\textbf{Obiettivo}: Accesso Python senza soluzione di continuità ad AstDyn.

\textbf{API pianificata}:
\begin{lstlisting}[language=Python]
import astdyn

# Crea elementi orbitali
elem = astdyn.KeplerianElements(
    a=2.7436, e=0.0624, i=11.74,
    Omega=339.86, omega=258.03, M=45.32,
    epoch=2460000.5
)

# Configura propagatore
eph = astdyn.SPICEEphemeris("de440.bsp")
forces = astdyn.PointMassGravity(eph, ["JUPITER", "SATURN"])
integrator = astdyn.RKF78(tolerance=1e-12)
prop = astdyn.Propagator(integrator, forces, eph)

# Propaga
state0 = elem.to_cartesian()
state60 = prop.propagate(state0, 2460060.5)

print(f"Posizione: {state60.position}")
print(f"Velocita': {state60.velocity}")

# Determinazione orbitale
observations = astdyn.read_mpc_file("observations.txt")
corrector = astdyn.DifferentialCorrector(prop)
result = corrector.solve(elem, observations)

print(f"Residuo RMS: {result.rms_residual} arcosecondi")
print(f"Convergenza: {result.converged}")
\end{lstlisting}

\textbf{Integrazione}: Notebook Jupyter, array NumPy, grafici matplotlib.

\subsection{Distribuzione del Pacchetto}

\textbf{Pianificato}:
\begin{itemize}
    \item Pacchetto PyPI: \texttt{pip install astdyn}
    \item Pacchetto Conda: \texttt{conda install -c conda-forge astdyn}
    \item Wheel pre-compilate per Linux, macOS, Windows
\end{itemize}

\section{Integrazione con Machine Learning}

\subsection{Modelli Surrogati con Reti Neurali}

\textbf{Direzione di ricerca}: Addestrare reti neurali per approssimare calcoli costosi.

\textbf{Applicazioni potenziali}:
\begin{enumerate}
    \item \textbf{Propagazione veloce}: NN approssima l'integratore per applicazioni in tempo reale
    \item \textbf{Rilevamento outlier}: ML identifica automaticamente osservazioni errate
    \item \textbf{Orbita iniziale}: NN fornisce una migliore stima IOD da osservazioni limitate
\end{enumerate}

\textbf{Concetto preliminare}:
\begin{lstlisting}[language=C++]
class NeuralPropagator : public IIntegrator {
public:
    NeuralPropagator(const std::string& model_file);
    
    // Usa NN per propagazione a breve termine
    CartesianState propagate(const CartesianState& initial, 
                            double dt) {
        if (dt < 10.0) {  // Usa NN per archi brevi
            return nn_predict(initial, dt);
        } else {  // Fallback a integrazione numerica
            return numerical_propagate(initial, dt);
        }
    }
    
private:
    NeuralNetwork model_;
};
\end{lstlisting}

\textbf{Sfida}: Garantire garanzie di accuratezza per uso scientifico.

\section{Osservazioni Avanzate}

\subsection{Osservazioni Radar}

\textbf{Pianificato}: Supporto per misure di distanza e velocità radiale.

\begin{lstlisting}[language=C++]
struct RadarObservation {
    double epoch;
    double range;           // km
    double range_rate;      // km/s
    double sigma_range;
    double sigma_range_rate;
    std::string station_code;
    
    Vector3d observer_position() const;
};
\end{lstlisting}

\textbf{Integrazione}: Aggiungi residui radar ai minimi quadrati:
\[
\chi^2 = \sum_i \frac{(\rho_i^{\text{obs}} - \rho_i^{\text{comp}})^2}{\sigma_{\rho,i}^2} + \frac{(\dot{\rho}_i^{\text{obs}} - \dot{\rho}_i^{\text{comp}})^2}{\sigma_{\dot{\rho},i}^2}
\]

\textbf{Beneficio}: Accuratezza in distanza di ordini di grandezza superiore rispetto all'ottico.

\subsection{Astrometria Gaia}

\textbf{Pianificato}: Supporto nativo per osservazioni del satellite Gaia.

\textbf{Funzionalità}:
\begin{itemize}
    \item Misure along-scan e across-scan
    \item Sistema di riferimento Gaia (ICRF3)
    \item Correzioni di parallasse
    \item Tempo luce e aberrazione a livello di $\mu$as
\end{itemize}

\section{Servizio Web / Deployment Cloud}

\subsection{API RESTful}

\textbf{Visione}: Servizio di determinazione orbitale basato su cloud.

\textbf{Endpoint pianificati}:
\begin{verbatim}
POST /api/v1/propagate
  Body: { "elements": {...}, "target_epoch": 2460100.5 }
  Returns: { "state": {...}, "elapsed_ms": 1.82 }

POST /api/v1/orbit_determination
  Body: { "observations": [...], "method": "differential_correction" }
  Returns: { "elements": {...}, "rms": 0.658, "iterations": 4 }

GET /api/v1/ephemeris?object=pompeja&start=2460000&end=2460100&step=1
  Returns: [ { "epoch": 2460000.5, "position": [...], ...}, ... ]
\end{verbatim}

\textbf{Stack tecnologico}:
\begin{itemize}
    \item Backend: Servizio C++ con wrapper REST
    \item Coda: Redis per gestione job
    \item Database: PostgreSQL per archiviazione risultati
    \item Container: Deployment Docker
\end{itemize}

\subsection{Interfaccia Web}

\textbf{Funzionalità}:
\begin{itemize}
    \item Caricamento file osservazioni MPC
    \item Visualizzazione interattiva orbita (3D)
    \item Download risultati (CSV, JSON, formato OrbFit)
    \item Confronto con JPL Horizons
    \item Interfaccia elaborazione batch
\end{itemize}

\section{Integrazione Pipeline Dati}

\subsection{Elaborazione Automatizzata di Survey}

\textbf{Obiettivo}: Processare automaticamente flussi dati LSST/Pan-STARRS.

\textbf{Pipeline}:
\begin{enumerate}
    \item Ingestione: Ricezione nuove osservazioni da survey
    \item Matching: Collegamento a oggetti noti o rilevamento di nuovi
    \item IOD: Orbita rapida per nuove rilevazioni
    \item Raffinamento: Correzione differenziale con dati d'archivio
    \item Pubblicazione: Aggiornamento database elementi orbitali
    \item Allerta: Segnalazione oggetti interessanti (NEA, orbite inusuali)
\end{enumerate}

\textbf{Scalabilità}: Elaborazione di oltre 1000 oggetti per notte.

\section{Modelli di Errore Migliorati}

\subsection{Stima Robusta}

\textbf{Attuale}: Minimi quadrati assume errori gaussiani.

\textbf{Pianificato}: Perdita di Huber e minimi quadrati ripesati iterativamente.

\begin{lstlisting}[language=C++]
class RobustDifferentialCorrector : public DifferentialCorrector {
public:
    Result solve(...) override {
        // Minimi quadrati iniziali
        auto result = standard_solve(...);
        
        // Iterazione robusta
        for (int iter = 0; iter < max_robust_iters; ++iter) {
            // Calcola pesi basati sui residui
            update_weights_huber(result.residuals);
            
            // Minimi quadrati pesati
            result = weighted_solve(...);
        }
        
        return result;
    }
};
\end{lstlisting}

\textbf{Beneficio}: Riduzione automatica del peso degli outlier.

\section{Supporto Multi-Piattaforma}

\subsection{Build WebAssembly}

\textbf{Obiettivo}: Eseguire AstDyn nel browser web.

\textbf{Casi d'uso}:
\begin{itemize}
    \item Strumenti educativi (calcolatore orbitale interattivo)
    \item Visualizzazione orbita lato client
    \item Nessun server richiesto per calcoli semplici
\end{itemize}

\textbf{Build}:
\begin{lstlisting}[language=bash]
emcc -O3 -s WASM=1 -s ALLOW_MEMORY_GROWTH=1 \
     astdyn.cpp -o astdyn.js
\end{lstlisting}

\subsection{Piattaforme Mobile}

\textbf{Pianificato}: Librerie native iOS e Android.

\textbf{Applicazioni}:
\begin{itemize}
    \item App di pianificazione osservativa
    \item Tracciamento satellitare in tempo reale
    \item App educative di astronomia
\end{itemize}

\section{Roadmap di Sviluppo}

\subsection{Versione 1.1 (Q2 2026)}

Funzionalità prioritarie:
\begin{itemize}
    \item Parallelizzazione OpenMP
    \item Binding Python (pybind11)
    \item Pressione di radiazione solare
    \item Supporto osservazioni radar
\end{itemize}

\subsection{Versione 1.2 (Q4 2026)}

Capacità estese:
\begin{itemize}
    \item Propagazione della covarianza
    \item Incertezza Monte Carlo
    \item Modello di degassamento cometario
    \item Integratore simplettico
\end{itemize}

\subsection{Versione 2.0 (2027)}

Miglioramenti maggiori:
\begin{itemize}
    \item Gestione incontri ravvicinati (regolarizzazione KS)
    \item Correzioni di relatività generale
    \item Accelerazione GPU (CUDA)
    \item Deployment servizio web
\end{itemize}

\section{Contributi della Comunità}

\subsection{Sviluppo Open Source}

AstDyn accoglie i contributi della comunità:

\textbf{Repository GitHub}: \url{https://github.com/user/astdyn}

\textbf{Aree di contribuzione}:
\begin{itemize}
    \item Nuovi integratori (Dormand-Prince, Radau)
    \item Parser aggiuntivi (JPL, SPICE SPK)
    \item Modelli di forza (forze di marea, GR)
    \item Miglioramenti alla documentazione
    \item Test case e validazione
    \item Ottimizzazioni delle prestazioni
\end{itemize}

\textbf{Linee guida}: Vedi CONTRIBUTING.md nel repository.

\subsection{Citazione di AstDyn}

Se utilizzate AstDyn nella ricerca, si prega di citare:

\begin{verbatim}
@software{astdyn2025,
  author = {Bigi, Michele and Contributors},
  title = {AstDyn: Modern C++ Library for Asteroid Orbit Determination},
  year = {2025},
  version = {1.0.0},
  url = {https://github.com/user/astdyn}
}
\end{verbatim}

\section{Direzioni di Ricerca}

\subsection{Algoritmi Innovativi}

Argomenti di ricerca futura:
\begin{enumerate}
    \item \textbf{IOD con deep learning}: Reti neurali per orbita iniziale da 2 osservazioni
    \item \textbf{Filtro di Kalman}: Determinazione orbitale sequenziale
    \item \textbf{Metodi bayesiani}: Soluzioni orbitali probabilistiche
    \item \textbf{Probabilità di collisione}: Monte Carlo veloce per analisi di congiunzione
    \item \textbf{Tracciamento multi-oggetto}: Determinazione orbitale simultanea per più asteroidi
\end{enumerate}

\subsection{Applicazioni Interdisciplinari}

Oltre gli asteroidi:
\begin{itemize}
    \item \textbf{Tracciamento detriti spaziali}: Determinazione orbite LEO/GEO
    \item \textbf{Asteroidi binari}: Dinamica di orbite mutue
    \item \textbf{Lune planetarie}: Determinazione orbite satellitari
    \item \textbf{Transiti esopianetari}: Analisi temporale
\end{itemize}

\section{Riepilogo}

Gli sviluppi pianificati per AstDyn includono:

\begin{enumerate}
    \item \textbf{Fisica}: Forze non gravitazionali, relatività, incontri ravvicinati
    \item \textbf{Algoritmi}: Propagazione incertezze, stima robusta, nuovi integratori
    \item \textbf{Prestazioni}: OpenMP, accelerazione GPU, deployment cloud
    \item \textbf{Interfacce}: Binding Python, servizio web, piattaforme mobile
    \item \textbf{Dati}: Osservazioni radar, astrometria Gaia, pipeline survey
    \item \textbf{Comunità}: Contributi open source, collaborazioni di ricerca
\end{enumerate}

Questi miglioramenti espanderanno le capacità di AstDyn mantenendo i principi di design fondamentali di accuratezza, affidabilità e facilità d'uso.

\vspace{1cm}

\textit{Contributi e suggerimenti sono benvenuti. Visita il repository GitHub per partecipare allo sviluppo di AstDyn.}


% Materia finale
\backmatter
\include{24_riferimenti}
\include{25_appendici}

\end{document}
