\chapter{Differential Correction}
\label{ch:differential_correction}

\section{Introduction}

\textbf{Differential correction} (DC) is the iterative least-squares refinement of an orbit using all available observations. It is the cornerstone of orbit determination.

\textbf{Input}: Initial orbit + observations

\textbf{Output}: Improved orbit + covariance matrix + residuals

\textbf{Method}: Weighted least squares minimizing O-C (observed minus computed) residuals.

\section{The Least Squares Problem}

\subsection{Observation Equation}

For observation $i$:

\begin{equation}
    \mathbf{o}_i = \mathbf{h}(\mathbf{y}_0, t_i) + \boldsymbol{\epsilon}_i
\end{equation}

where:
\begin{itemize}
    \item $\mathbf{o}_i$: Observed value (e.g., RA, Dec)
    \item $\mathbf{h}$: Observation model (coordinate transformation)
    \item $\mathbf{y}_0$: State at epoch $t_0$
    \item $\boldsymbol{\epsilon}_i \sim \mathcal{N}(0, \mathbf{W}_i^{-1})$: Measurement error
\end{itemize}

\subsection{Linearization}

Linearize around current estimate $\mathbf{y}_0^{(k)}$:

\begin{equation}
    \mathbf{o}_i - \mathbf{c}_i = \mathbf{H}_i \Delta\mathbf{y}_0 + \boldsymbol{\epsilon}_i
\end{equation}

where:
\begin{itemize}
    \item $\mathbf{c}_i = \mathbf{h}(\mathbf{y}_0^{(k)}, t_i)$: Computed value
    \item $\mathbf{H}_i = \frac{\partial \mathbf{h}}{\partial \mathbf{y}_0}$: Design matrix (observation partials)
    \item $\Delta\mathbf{y}_0 = \mathbf{y}_0 - \mathbf{y}_0^{(k)}$: State correction
\end{itemize}

\subsection{Normal Equations}

Minimize weighted sum of squared residuals:

\begin{equation}
    \chi^2 = \sum_{i=1}^m (\mathbf{o}_i - \mathbf{c}_i - \mathbf{H}_i \Delta\mathbf{y}_0)^T \mathbf{W}_i (\mathbf{o}_i - \mathbf{c}_i - \mathbf{H}_i \Delta\mathbf{y}_0)
\end{equation}

Solution:

\begin{equation}
    (\mathbf{H}^T \mathbf{W} \mathbf{H}) \Delta\mathbf{y}_0 = \mathbf{H}^T \mathbf{W} (\mathbf{o} - \mathbf{c})
\end{equation}

Define:
\begin{align}
    \mathbf{N} &= \mathbf{H}^T \mathbf{W} \mathbf{H} \quad \text{(normal matrix)} \\
    \mathbf{b} &= \mathbf{H}^T \mathbf{W} (\mathbf{o} - \mathbf{c}) \quad \text{(right-hand side)}
\end{align}

Solution: $\mathbf{N} \Delta\mathbf{y}_0 = \mathbf{b}$

Covariance: $\mathbf{C} = \mathbf{N}^{-1}$

\section{Computing Observation Partials}

\subsection{Chain Rule with STM}

For RA/Dec observations at time $t_i$:

\begin{equation}
    \mathbf{H}_i = \frac{\partial (\alpha, \delta)}{\partial \mathbf{y}_0} = \frac{\partial (\alpha, \delta)}{\partial \mathbf{y}(t_i)} \frac{\partial \mathbf{y}(t_i)}{\partial \mathbf{y}_0}
\end{equation}

where $\Phi(t_i, t_0) = \frac{\partial \mathbf{y}(t_i)}{\partial \mathbf{y}_0}$ is the state transition matrix (Chapter 10).

\subsection{Geometric Partials}

From topocentric position $\boldsymbol{\rho} = \mathbf{r} - \mathbf{R}$:

\begin{align}
    \alpha &= \arctan2(\rho_y, \rho_x) \\
    \delta &= \arcsin(\rho_z / \rho)
\end{align}

Partials:

\begin{align}
    \frac{\partial \alpha}{\partial \rho_x} &= -\frac{\rho_y}{\rho_x^2 + \rho_y^2} \\
    \frac{\partial \alpha}{\partial \rho_y} &= \frac{\rho_x}{\rho_x^2 + \rho_y^2} \\
    \frac{\partial \alpha}{\partial \rho_z} &= 0 \\
    \frac{\partial \delta}{\partial \rho_x} &= -\frac{\rho_x \rho_z}{\rho^2 \sqrt{\rho_x^2 + \rho_y^2}} \\
    \frac{\partial \delta}{\partial \rho_y} &= -\frac{\rho_y \rho_z}{\rho^2 \sqrt{\rho_x^2 + \rho_y^2}} \\
    \frac{\partial \delta}{\partial \rho_z} &= \frac{\sqrt{\rho_x^2 + \rho_y^2}}{\rho^2}
\end{align}

\subsection{Frame Rotation and Coordinate Systems}

A critical aspect of practical implementation is handling different coordinate systems.
Typically, numerical integration and STM propagation are performed in the \textbf{Heliocentric Ecliptic J2000} frame (to align with planetary ephemerides like VSOP87), while observations are reported in the \textbf{Topocentric Equatorial J2000} frame (RA/Dec).

Therefore, the chain rule must include a rotation matrix $\mathbf{R}_{\text{ecl}\to\text{eq}}$:

\begin{equation}
    \mathbf{H}_i = \frac{\partial (\alpha, \delta)}{\partial \mathbf{r}_{\text{eq}}} \cdot \mathbf{R}_{\text{ecl}\to\text{eq}} \cdot \Phi_{\text{ecl}}(t_i, t_0)
\end{equation}

where:
\begin{itemize}
    \item $\frac{\partial (\alpha, \delta)}{\partial \mathbf{r}_{\text{eq}}}$ are the geometric partials in the equatorial frame.
    \item $\mathbf{R}_{\text{ecl}\to\text{eq}}$ is the rotation matrix for the obliquity of the ecliptic ($\epsilon \approx 23.44^\circ$).
    \item $\Phi_{\text{ecl}}(t_i, t_0)$ is the STM in the ecliptic frame.
\end{itemize}

Neglecting this rotation when computing partials will lead to fit divergence, as the gradient direction will be incorrect.

\subsection{Light-Time Correction}

The state $\mathbf{y}(t_i)$ used in the observation equation is actually the state at the retarded time $t_i - \tau$, where $\tau$ is the light travel time. The partial derivatives should technically account for this time shift, but for main belt asteroids, the approximation $\frac{\partial \mathbf{y}(t_i-\tau)}{\partial \mathbf{y}_0} \approx \Phi(t_i, t_0)$ is usually sufficient.

\subsection{Full Partials}

Combine geometric partials, rotation, and $\Phi$:

\begin{equation}
    \mathbf{H}_i = \begin{bmatrix}
        \frac{\partial \alpha}{\partial x_{\text{eq}}} & \frac{\partial \alpha}{\partial y_{\text{eq}}} & \frac{\partial \alpha}{\partial z_{\text{eq}}} & 0 & 0 & 0 \\
        \frac{\partial \delta}{\partial x_{\text{eq}}} & \frac{\partial \delta}{\partial y_{\text{eq}}} & \frac{\partial \delta}{\partial z_{\text{eq}}} & 0 & 0 & 0
    \end{bmatrix} 
    \begin{bmatrix}
        \mathbf{R} & \mathbf{0} \\
        \mathbf{0} & \mathbf{R}
    \end{bmatrix}
    \Phi_{\text{ecl}}(t_i, t_0)
\end{equation}

Note: RA/Dec depend only on position, not velocity, in geometric partials. Velocity affects observations through propagation ($\Phi$).

\section{Algorithm}

\textbf{Input}: Initial orbit $\mathbf{y}_0^{(0)}$, observations $\{(\mathbf{o}_i, t_i, \mathbf{W}_i)\}$

\textbf{Iterate}:
\begin{enumerate}
    \item For each observation $i$:
    \begin{enumerate}
        \item Propagate to $t_i$ with STM: $[\mathbf{y}(t_i), \Phi(t_i, t_0)]$
        \item Compute prediction $\mathbf{c}_i = \mathbf{h}(\mathbf{y}(t_i))$
        \item Compute geometric partials
        \item Compute full partials $\mathbf{H}_i$ using STM
    \end{enumerate}
    \item Form normal matrix: $\mathbf{N} = \sum_i \mathbf{H}_i^T \mathbf{W}_i \mathbf{H}_i$
    \item Form RHS: $\mathbf{b} = \sum_i \mathbf{H}_i^T \mathbf{W}_i (\mathbf{o}_i - \mathbf{c}_i)$
    \item Solve: $\mathbf{N} \Delta\mathbf{y}_0 = \mathbf{b}$
    \item Update: $\mathbf{y}_0^{(k+1)} = \mathbf{y}_0^{(k)} + \Delta\mathbf{y}_0$
    \item Compute RMS: $\text{RMS} = \sqrt{\frac{1}{m-n} \sum_i w_i r_i^2}$ where $r_i = \mathbf{o}_i - \mathbf{c}_i$
    \item Check convergence: $|\Delta\mathbf{y}_0| < \epsilon$ and $|\Delta\text{RMS}| < \epsilon_{\text{RMS}}$
\end{enumerate}

\textbf{Output}: Converged state $\mathbf{y}_0^*$, covariance $\mathbf{C} = \mathbf{N}^{-1}$, residuals

\section{Convergence Criteria}

\subsection{State Correction}

\begin{equation}
    ||\Delta\mathbf{y}_0|| < 10^{-8} \text{ AU, AU/day}
\end{equation}

\subsection{RMS Change}

\begin{equation}
    \frac{|\text{RMS}^{(k+1)} - \text{RMS}^{(k)}|}{\text{RMS}^{(k)}} < 10^{-6}
\end{equation}

\subsection{Maximum Iterations}

Typically converges in 3-10 iterations. If not converged after 20 iterations, suspect:
\begin{itemize}
    \item Poor initial orbit
    \item Bad observations (outliers)
    \item Model inadequacy (missing perturbations)
\end{itemize}

\section{Weighting Strategy}

\subsection{Empirical Weights}

For RA/Dec observations:

\begin{equation}
    w_{\alpha,i} = \frac{1}{\sigma_{\alpha,i}^2}, \quad w_{\delta,i} = \frac{1}{\sigma_{\delta,i}^2}
\end{equation}

Typical $\sigma$:
\begin{itemize}
    \item Modern CCD (Gaia-calibrated): 0.1"
    \item Amateur CCD: 0.5"
    \item Historical photographic: 1-2"
\end{itemize}

\subsection{Robust Weighting}

Downweight outliers using Huber weights:

\begin{equation}
    w_i' = \begin{cases}
        w_i & \text{if } |r_i| < k\sigma \\
        w_i \frac{k\sigma}{|r_i|} & \text{if } |r_i| \ge k\sigma
    \end{cases}
\end{equation}

where $k = 2.5$ (typical).

\section{Covariance Matrix}

\subsection{Formal Uncertainty}

From normal matrix:

\begin{equation}
    \mathbf{C} = \mathbf{N}^{-1} = (\mathbf{H}^T \mathbf{W} \mathbf{H})^{-1}
\end{equation}

Diagonal elements: $\sigma_i = \sqrt{C_{ii}}$

\textbf{Example} (asteroid with 100 observations over 30 days):
\begin{itemize}
    \item $\sigma_x \sim 10^{-7}$ AU (15 km)
    \item $\sigma_v \sim 10^{-9}$ AU/day (1.7 mm/s)
\end{itemize}

\subsection{Correlation}

Off-diagonal elements show parameter correlations:

\begin{equation}
    \rho_{ij} = \frac{C_{ij}}{\sqrt{C_{ii} C_{jj}}}
\end{equation}

Strong correlations (e.g., $\rho_{xy} > 0.9$) indicate observational geometry issues.

\subsection{Propagated Uncertainty}

At time $t$:

\begin{equation}
    \mathbf{C}(t) = \Phi(t, t_0) \mathbf{C}(t_0) \Phi(t, t_0)^T
\end{equation}

Uncertainty grows with time. For short-arc solutions, $\sigma$ can increase exponentially.

\section{Implementation}

\begin{lstlisting}[language=C++,caption={Differential correction implementation}]
struct DCResult {
    Vector6d state;
    Matrix6d covariance;
    double rms;
    int iterations;
    std::vector<double> residuals;
};

DCResult differential_correction(
    const Vector6d& initial_state,
    double epoch,
    const std::vector<Observation>& observations,
    const ForceModel& forces,
    const EphemerisInterface& ephemeris,
    int max_iterations = 20,
    double tol = 1e-8)
{
    Vector6d y0 = initial_state;
    double prev_rms = 1e10;
    
    for (int iter = 0; iter < max_iterations; ++iter) {
        // Accumulate normal matrix and RHS
        Matrix6d N = Matrix6d::Zero();
        Vector6d b = Vector6d::Zero();
        double chi2 = 0.0;
        std::vector<double> residuals;
        
        for (const auto& obs : observations) {
            // Propagate with STM
            auto [y_obs, Phi] = propagate_with_stm(y0, epoch, obs.epoch, forces);
            
            // Predict observation
            Vector2d computed = predict_observation(y_obs, obs.epoch, obs.obs_code, ephemeris);
            
            // Residual (O-C)
            Vector2d residual;
            residual(0) = (obs.ra - computed(0)) * cos(obs.dec);  // RA cos(Dec)
            residual(1) = obs.dec - computed(1);  // Dec
            
            residuals.push_back(residual.norm() * RAD_TO_ARCSEC);
            
            // Geometric partials
            Matrix<double, 2, 3> geom_partials = compute_ra_dec_partials(y_obs, obs, ephemeris);
            
            // Full partials via STM
            Matrix<double, 2, 6> H;
            H.block<2, 3>(0, 0) = geom_partials;
            H.block<2, 3>(0, 3).setZero();
            H = H * Phi;  // Chain rule
            
            // Weights
            double w_ra = 1.0 / (obs.sigma_ra * obs.sigma_ra);
            double w_dec = 1.0 / (obs.sigma_dec * obs.sigma_dec);
            Matrix2d W = Vector2d(w_ra, w_dec).asDiagonal();
            
            // Accumulate normal equations
            N += H.transpose() * W * H;
            b += H.transpose() * W * residual;
            chi2 += residual.transpose() * W * residual;
        }
        
        // Solve normal equations
        Vector6d delta_y0 = N.ldlt().solve(b);
        
        // Update state
        y0 += delta_y0;
        
        // Compute RMS
        int dof = 2 * observations.size() - 6;  // degrees of freedom
        double rms = sqrt(chi2 / dof) * RAD_TO_ARCSEC;
        
        // Check convergence
        if (delta_y0.norm() < tol && abs(rms - prev_rms) < 1e-6) {
            Matrix6d covariance = N.inverse();
            return {y0, covariance, rms, iter + 1, residuals};
        }
        
        prev_rms = rms;
    }
    
    throw std::runtime_error("DC did not converge");
}
\end{lstlisting}

\section{Example: Asteroid 203 Pompeja}

\subsection{Problem Setup}

\begin{itemize}
    \item Object: 203 Pompeja (Main Belt asteroid)
    \item Observations: 100 RA/Dec measurements
    \item Time span: 60 days
    \item Observatory: 500 (geocentric), F51 (Pan-STARRS)
    \item Initial orbit: From JPL Horizons
\end{itemize}

\subsection{Results}

\begin{lstlisting}[language=C++,caption={Running DC on Pompeja}]
// Load observations from MPC format file
std::vector<Observation> obs = load_mpc_observations("pompeja.obs");
std::cout << "Loaded " << obs.size() << " observations\n";

// Initial orbit from Horizons
Vector6d y0_initial = /* ... from JPL ... */;
double epoch = 2460000.5;  // JD

// Force model
auto forces = std::make_shared<ForceModel>();
forces->add_perturbation(std::make_shared<SunGravity>());
forces->add_perturbation(std::make_shared<JupiterPerturbation>());
forces->add_perturbation(std::make_shared<SaturnPerturbation>());

// Ephemeris
SpiceInterface spice;
spice.load_kernel("de440.bsp");

// Run differential correction
try {
    auto result = differential_correction(y0_initial, epoch, obs, *forces, spice);
    
    std::cout << "Converged in " << result.iterations << " iterations\n";
    std::cout << "RMS = " << result.rms << " arcsec\n";
    
    // Print orbital elements
    OrbitalElements elem = OrbitalElements::from_cartesian(result.state, epoch);
    std::cout << "\nImproved orbit:\n";
    std::cout << "a = " << elem.a << " +/- " << sqrt(result.covariance(0,0)) << " AU\n";
    std::cout << "e = " << elem.e << " +/- " << sqrt(result.covariance(1,1)) << "\n";
    std::cout << "i = " << elem.i * RAD_TO_DEG << " deg\n";
    
    // Largest residuals
    std::sort(result.residuals.begin(), result.residuals.end(), std::greater<>());
    std::cout << "\nTop 5 residuals:\n";
    for (int i = 0; i < 5; ++i) {
        std::cout << i+1 << ". " << result.residuals[i] << " arcsec\n";
    }
    
} catch (const std::exception& e) {
    std::cerr << "Error: " << e.what() << "\n";
}
\end{lstlisting}

\textbf{Typical output}:
\begin{verbatim}
Loaded 100 observations
Converged in 5 iterations
RMS = 0.658 arcsec

Improved orbit:
a = 2.7436 +/- 0.000001 AU
e = 0.0624 +/- 0.000005
i = 11.743 deg

Top 5 residuals:
1. 2.34 arcsec
2. 1.98 arcsec
3. 1.76 arcsec
4. 1.65 arcsec
5. 1.54 arcsec
\end{verbatim}

\subsection{Interpretation}

\begin{itemize}
    \item \textbf{RMS = 0.658"}: Excellent fit, consistent with CCD astrometry precision
    \item \textbf{5 iterations}: Rapid convergence indicates good initial orbit
    \item \textbf{$\sigma_a = 10^{-6}$ AU}: Semimajor axis determined to ~150 km
    \item \textbf{Top residuals $<$2.5"}: No obvious outliers
    \item \textbf{Covariance}: Formal uncertainty, propagate for ephemeris error
\end{itemize}

\section{Troubleshooting}

\subsection{Non-Convergence}

\textbf{Symptoms}: RMS oscillates or increases.

\textbf{Causes}:
\begin{enumerate}
    \item Poor initial orbit (too far from truth)
    \item Outliers dominating fit
    \item Force model inadequate
    \item Numerical issues (ill-conditioned normal matrix)
\end{enumerate}

\textbf{Solutions}:
\begin{itemize}
    \item Improve IOD
    \item Enable robust weighting
    \item Add missing perturbations
    \item Regularize normal matrix
\end{itemize}

\subsection{Large RMS}

\textbf{Symptoms}: RMS $>$ 2" for modern observations.

\textbf{Causes}:
\begin{itemize}
    \item Systematic errors in observations
    \item Wrong observatory coordinates
    \item Timing errors
    \item Missing perturbations (e.g., close encounter)
\end{itemize}

\textbf{Diagnosis}: Plot residuals vs. time, magnitude, observatory.

\subsection{Small Residuals but Wrong Orbit}

\textbf{Symptoms}: RMS $<$ 0.5" but ephemeris predictions fail.

\textbf{Cause}: Short arc + degeneracy. Many orbits fit equally well over short spans.

\textbf{Solution}: Acquire observations over longer arc (>30 days for main belt, >7 days for NEA).

\section{Summary}

Key points about differential correction:

\begin{enumerate}
    \item \textbf{Least squares} minimizes weighted sum of O-C squared residuals
    \item \textbf{Normal equations} $\mathbf{N}\Delta\mathbf{y}_0 = \mathbf{b}$ solved iteratively
    \item \textbf{Observation partials} computed via chain rule with STM
    \item \textbf{Geometric partials} relate RA/Dec to topocentric position
    \item \textbf{Convergence} typically in 3-10 iterations
    \item \textbf{Covariance matrix} $\mathbf{C} = \mathbf{N}^{-1}$ gives formal uncertainty
    \item \textbf{RMS} indicates fit quality; target $<$1" for modern CCD
    \item \textbf{Robust weighting} downweights outliers
    \item \textbf{Pompeja example} demonstrates complete workflow
\end{enumerate}

Next chapter covers residual analysis for quality assessment and outlier detection.

